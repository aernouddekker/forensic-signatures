% Forensic Signatures Consistent with Bandwidth-Limited Recovery Dynamics in Qubits and LIGO Glitches
\documentclass[11pt,a4paper]{article}

% ============================================================
% PACKAGES
% ============================================================
\usepackage[utf8]{inputenc}
\usepackage[T1]{fontenc}
\usepackage{lmodern}
\usepackage{microtype}  % Microtypographic refinements
\usepackage{amsmath,amssymb,amsfonts}
\usepackage{graphicx}
\usepackage{xcolor}
\usepackage{booktabs}
\usepackage[font=footnotesize,labelfont=bf]{caption}
\usepackage{subcaption}
\usepackage[margin=1in]{geometry}
\usepackage{float}
\usepackage[numbers]{natbib}
\usepackage{hyperref}  % Load after natbib for proper citation links

% ============================================================
% HYPERREF SETUP
% ============================================================
\hypersetup{
    colorlinks=true,
    linkcolor=black,
    citecolor=black,
    urlcolor=blue!60!black,
    pdftitle={Forensic Signatures Consistent with Bandwidth-Limited Recovery Dynamics in Qubits and LIGO Glitches},
    pdfauthor={Aernoud Dekker}
}

% Make PDF bookmarks/metadata safe for our custom commands
\pdfstringdefDisableCommands{%
  \def\textbar{|}%
  \def\NA{NA}%
}

% ============================================================
% CUSTOM COMMANDS
% ============================================================
\providecommand{\BLQC}{BLQC}
\providecommand{\Ceff}{\ensuremath{C_{\text{eff}}}}
\providecommand{\tauSK}{\ensuremath{\tau_{\text{SK}}}}

% Canonical results macros (prevents number drift)
% ===== BEGIN INLINED: results_macros.tex =====
% results_macros.tex
% Canonical results - use these everywhere to prevent number drift
% Auto-generated by generate_macros.py on 2025-12-19 19:05
% DO NOT EDIT MANUALLY - regenerate with: python scripts/generate_macros.py

% === Provenance (for reproducibility audit) ===
% Safety defaults for provenance macros
\providecommand{\ResultsRunDate}{UNKNOWN}
\providecommand{\ResultsRNGSeed}{UNKNOWN}
\providecommand{\ResultsPipelineVersion}{UNKNOWN}
\providecommand{\ResultsGitHash}{UNKNOWN}
\providecommand{\ResultsBanner}{UNKNOWN}
% Actual values
\renewcommand{\ResultsRunDate}{2025-12-19}
\renewcommand{\ResultsRNGSeed}{42}
\renewcommand{\ResultsPipelineVersion}{1.0}
\renewcommand{\ResultsGitHash}{018acde}

% One-line provenance banner for appendix
\renewcommand{\ResultsBanner}{Pipeline v1.0 \textbar\ seed 42 \textbar\ 018acde \textbar\ 2025-12-19}

% === Formatting helpers (prevent style drift) ===
% Math-safe placeholder for missing values (won't italicize in math mode)
\providecommand{\NA}{\ensuremath{\mathrm{NA}}}
% Other formatting helpers
\providecommand{\Pct}[1]{#1\%}
\providecommand{\CountPct}[2]{#1 (#2\%)}
\providecommand{\AUCfmt}[3]{#1 [#2, #3]}
\providecommand{\CIfmt}[2]{[#1, #2]}

% === Safety net defaults (use \providecommand so missing values show NA) ===
% These are overwritten below if data exists; NA indicates missing pipeline output
\providecommand{\NMcEwen}{\NA}
\providecommand{\McEwenStableDelayed}{\NA}
\providecommand{\McEwenStableDelayedPct}{\NA}
\providecommand{\McEwenStableDelayedCILo}{\NA}
\providecommand{\McEwenStableDelayedCIHi}{\NA}
\providecommand{\McEwenStableFast}{\NA}
\providecommand{\McEwenStableFastPct}{\NA}
\providecommand{\McEwenFlip}{\NA}
\providecommand{\McEwenFlipPct}{\NA}
\providecommand{\McEwenCurvatureP}{\NA}
\providecommand{\McEwenTauMedian}{\NA}
\providecommand{\McEwenTauIQRLo}{\NA}
\providecommand{\McEwenTauIQRHi}{\NA}
\providecommand{\McEwenRobustN}{\NA}
\providecommand{\McEwenRobustConfigs}{\NA}
\providecommand{\McEwenStableIOFEvidMed}{\NA}
\providecommand{\McEwenStableIOFEvidIQRLo}{\NA}
\providecommand{\McEwenStableIOFEvidIQRHi}{\NA}
\providecommand{\McEwenStableSTDEvidMed}{\NA}
\providecommand{\McEwenStableSTDEvidIQRLo}{\NA}
\providecommand{\McEwenStableSTDEvidIQRHi}{\NA}
\providecommand{\McEwenFlipEvidMed}{\NA}
\providecommand{\McEwenFlipEvidIQRLo}{\NA}
\providecommand{\McEwenFlipEvidIQRHi}{\NA}
\providecommand{\McEwenStableIOFAboveFour}{\NA}
\providecommand{\McEwenStableIOFAboveTen}{\NA}
\providecommand{\McEwenFlipAboveFour}{\NA}
\providecommand{\NLIGOOK}{\NA}
\providecommand{\NLIGOStable}{\NA}
\providecommand{\LIGOStableDelayed}{\NA}
\providecommand{\LIGOStableDelayedPct}{\NA}
\providecommand{\LIGOStableFast}{\NA}
\providecommand{\LIGOStableFastPct}{\NA}
\providecommand{\LIGOFlip}{\NA}
\providecommand{\LIGOFlipPct}{\NA}
\providecommand{\LIGOFailed}{\NA}
\providecommand{\LIGOFailedPct}{\NA}
\providecommand{\LIGOCurvatureAUC}{\NA}
\providecommand{\LIGOCurvatureAUClo}{\NA}
\providecommand{\LIGOCurvatureAUChi}{\NA}
\providecommand{\LIGOCliffsDelta}{\NA}
\providecommand{\LIGOCurvDelayedMedian}{\NA}
\providecommand{\LIGOCurvDelayedIQRLo}{\NA}
\providecommand{\LIGOCurvDelayedIQRHi}{\NA}
\providecommand{\LIGOCurvFastMedian}{\NA}
\providecommand{\LIGOCurvFastIQRLo}{\NA}
\providecommand{\LIGOCurvFastIQRHi}{\NA}
\providecommand{\LIGOBetaB}{\NA}
\providecommand{\LIGOBetaBLo}{\NA}
\providecommand{\LIGOBetaBHi}{\NA}
\providecommand{\LIGOBetaSNR}{\NA}
\providecommand{\LIGODeltaBIC}{\NA}
\providecommand{\LIGODeltaBICLo}{\NA}
\providecommand{\LIGODeltaBICHi}{\NA}
\providecommand{\LIGOBetaBOnly}{\NA}
\providecommand{\LIGOBetaBOnlyLo}{\NA}
\providecommand{\LIGOBetaBOnlyHi}{\NA}
\providecommand{\LIGODeltaBICBOnly}{\NA}
\providecommand{\LIGODeltaBICBOnlyLo}{\NA}
\providecommand{\LIGODeltaBICBOnlyHi}{\NA}
\providecommand{\LIGOBetaSNROnly}{\NA}
\providecommand{\LIGOThreshOneDelayedN}{\NA}
\providecommand{\LIGOThreshOneDelayedPct}{\NA}
\providecommand{\LIGOThreshOneFastN}{\NA}
\providecommand{\LIGOThreshOneFastPct}{\NA}
\providecommand{\LIGOThreshOneFlipN}{\NA}
\providecommand{\LIGOThreshOneFlipPct}{\NA}
\providecommand{\LIGOThreshOneAUC}{\NA}
\providecommand{\LIGOThreshFourDelayedN}{\NA}
\providecommand{\LIGOThreshFourDelayedPct}{\NA}
\providecommand{\LIGOThreshFourFastN}{\NA}
\providecommand{\LIGOThreshFourFastPct}{\NA}
\providecommand{\LIGOThreshFourFlipN}{\NA}
\providecommand{\LIGOThreshFourFlipPct}{\NA}
\providecommand{\LIGOThreshFourAUC}{\NA}
\providecommand{\LIGOFlipDeterminateN}{\NA}
\providecommand{\LIGOFlipDeterminatePct}{\NA}
\providecommand{\LIGOFlipUncertainN}{\NA}
\providecommand{\LIGOFlipUncertainPct}{\NA}
\providecommand{\LIGOFlipShortDelayedPct}{\NA}
\providecommand{\LIGOFlipLongDelayedPct}{\NA}
\providecommand{\LIGODip}{\NA}
\providecommand{\LIGODipP}{\NA}
\providecommand{\LIGODipPDelayed}{\NA}
\providecommand{\LIGODipPFast}{\NA}
\providecommand{\LIGOMWCurvPExp}{\NA}
\providecommand{\LIGODipUnlabeledN}{\NA}
\providecommand{\LIGODipUnlabeled}{\NA}
\providecommand{\LIGODipUnlabeledP}{\NA}
\providecommand{\LIGODipUnlabeledWinsP}{\NA}
\providecommand{\LIGODipUnlabeledTrimP}{\NA}
\providecommand{\LIGODipStableN}{\NA}
\providecommand{\LIGODipStableP}{\NA}
\providecommand{\LIGODipStableTrimP}{\NA}
\providecommand{\LIGOWindowN}{\NA}
\providecommand{\LIGOWindowClusters}{\NA}
\providecommand{\LIGOWindowBetaB}{\NA}
\providecommand{\LIGOWindowZB}{\NA}
\providecommand{\LIGOWindowPB}{\NA}
\providecommand{\LIGOWindowBetaSNR}{\NA}
\providecommand{\LIGOWindowPSNR}{\NA}
\providecommand{\LIGOGMMAgreement}{\NA}
\providecommand{\LIGOGMMARI}{\NA}
\providecommand{\LIGOGMMNMI}{\NA}
\providecommand{\LIGOGMMBoundary}{\NA}
\providecommand{\LIGORejectN}{\NA}
\providecommand{\LIGORejectKS}{\NA}
\providecommand{\LIGORejectKSP}{\NA}
\providecommand{\LIGORejectDelayedN}{\NA}
\providecommand{\LIGORejectFastN}{\NA}
\providecommand{\LIGORejectUncertainN}{\NA}
\providecommand{\LIGORejectDeterminateN}{\NA}
\providecommand{\LIGORejectAUC}{\NA}
\providecommand{\LIGOAltLikelihoodAgreement}{\NA}
\providecommand{\LIGOCurvSweepTenAUC}{\NA}
\providecommand{\LIGOCurvSweepTenCILo}{\NA}
\providecommand{\LIGOCurvSweepTenCIHi}{\NA}
\providecommand{\LIGOCurvSweepTenSign}{\NA}
\providecommand{\LIGOCurvSweepTwentyAUC}{\NA}
\providecommand{\LIGOCurvSweepTwentyCILo}{\NA}
\providecommand{\LIGOCurvSweepTwentyCIHi}{\NA}
\providecommand{\LIGOCurvSweepTwentySign}{\NA}
\providecommand{\LIGOCurvSweepThirtyAUC}{\NA}
\providecommand{\LIGOCurvSweepThirtyCILo}{\NA}
\providecommand{\LIGOCurvSweepThirtyCIHi}{\NA}
\providecommand{\LIGOCurvSweepThirtySign}{\NA}
\providecommand{\LIGOCurvSpearmanTenTwenty}{\NA}
\providecommand{\LIGOCurvSpearmanTwentyThirty}{\NA}
\providecommand{\LIGOAllGlitches}{\NA}
\providecommand{\LIGOExtremelyLoudTotal}{\NA}
\providecommand{\LIGOAfterCuts}{\NA}
\providecommand{\LIGODedupUnique}{\NA}
\providecommand{\LIGODedupFrom}{\NA}
\providecommand{\LIGONullN}{\NA}
\providecommand{\LIGONullMislabelRate}{\NA}
\providecommand{\LIGONullMislabelCILo}{\NA}
\providecommand{\LIGONullMislabelCIHi}{\NA}
\providecommand{\LIGONullMislabeledN}{\NA}
\providecommand{\LIGONullAmbiguousN}{\NA}
\providecommand{\LIGONullAUCAssigned}{\NA}
\providecommand{\LIGONullAUCRandom}{\NA}
\providecommand{\LIGONullLateWindow}{\NA}
\providecommand{\LIGONullEarlyWindow}{\NA}
\providecommand{\LIGOMixedNTotal}{\NA}
\providecommand{\LIGOMixedNValid}{\NA}
\providecommand{\LIGOMixedNAmbiguous}{\NA}
\providecommand{\LIGOMixedAccuracy}{\NA}
\providecommand{\LIGOMixedAUCTrue}{\NA}
\providecommand{\LIGOMixedAUCAssigned}{\NA}
\providecommand{\LIGOMixedCMFF}{\NA}
\providecommand{\LIGOMixedCMFD}{\NA}
\providecommand{\LIGOMixedCMDF}{\NA}
\providecommand{\LIGOMixedCMDD}{\NA}
\providecommand{\LIGOSweepFixedMax}{\NA}
\providecommand{\LIGOSweepFixedMaxCIHi}{\NA}
\providecommand{\LIGOSweepWindowMax}{\NA}
\providecommand{\LIGOSweepWindowMaxCIHi}{\NA}
\providecommand{\LIGOSweepWorstWindow}{\NA}
\providecommand{\LIGOSweepOverallMax}{\NA}
\providecommand{\LIGOSweepOverallMaxCIHi}{\NA}
\providecommand{\LIGOSplicedAUCUnspliced}{\NA}
\providecommand{\LIGOSplicedAUCTrue}{\NA}
\providecommand{\LIGOSplicedAUCMean}{\NA}
\providecommand{\LIGOSplicedAUCStd}{\NA}
\providecommand{\LIGOSplicedAUCDrop}{\NA}
\providecommand{\LIGOSeedSweepDelayedMin}{\NA}
\providecommand{\LIGOSeedSweepDelayedMax}{\NA}
\providecommand{\LIGOSeedSweepDelayedSD}{\NA}
\providecommand{\LIGOSeedSweepAUCMin}{\NA}
\providecommand{\LIGOSeedSweepAUCMax}{\NA}
\providecommand{\LIGOSeedSweepGMMAgreementMin}{\NA}
\providecommand{\LIGOSeedSweepGMMAgreementMax}{\NA}

% === McEwen / Google Sycamore (telemetry) ===
\renewcommand{\NMcEwen}{230}
\renewcommand{\McEwenStableDelayed}{15}
\renewcommand{\McEwenStableDelayedPct}{6.5}
\renewcommand{\McEwenStableDelayedCILo}{4.0}
\renewcommand{\McEwenStableDelayedCIHi}{10.5}
\renewcommand{\McEwenStableFast}{68}
\renewcommand{\McEwenStableFastPct}{29.6}
\renewcommand{\McEwenFlip}{147}
\renewcommand{\McEwenFlipPct}{63.9}
% Curvature discrimination (from stability_diagnostics.json)
\renewcommand{\McEwenCurvatureP}{0.44}  % Mann-Whitney p-value
% Recovery timescale (from mcewan_tau_stats.json)
\renewcommand{\McEwenTauMedian}{25.9}  % ms
\renewcommand{\McEwenTauIQRLo}{24.2}
\renewcommand{\McEwenTauIQRHi}{28.1}
% Key: delay (D) separates, curvature (b) does NOT

% === McEwen / Google robustness sweep (evidence strength) ===
\renewcommand{\McEwenRobustN}{3447}
\renewcommand{\McEwenRobustConfigs}{15}
\renewcommand{\McEwenStableIOFEvidMed}{4.99}
\renewcommand{\McEwenStableIOFEvidIQRLo}{3.62}
\renewcommand{\McEwenStableIOFEvidIQRHi}{8.08}
\renewcommand{\McEwenStableSTDEvidMed}{2.02}
\renewcommand{\McEwenStableSTDEvidIQRLo}{1.31}
\renewcommand{\McEwenStableSTDEvidIQRHi}{3.51}
\renewcommand{\McEwenFlipEvidMed}{1.97}
\renewcommand{\McEwenFlipEvidIQRLo}{0.82}
\renewcommand{\McEwenFlipEvidIQRHi}{3.22}

% Evidence threshold rates (from stability_diagnostics.json)
\renewcommand{\McEwenStableIOFAboveFour}{68}
\renewcommand{\McEwenStableIOFAboveTen}{15}
\renewcommand{\McEwenFlipAboveFour}{19}

% === LIGO Funnel Counts (Gravity Spy H1 O3a catalog) ===
\renewcommand{\LIGOAllGlitches}{80,763}
\renewcommand{\LIGOExtremelyLoudTotal}{10,684}
\renewcommand{\LIGOAfterCuts}{10,099}
\renewcommand{\LIGODedupUnique}{4,113}
\renewcommand{\LIGODedupFrom}{10,099}

% === LIGO Extremely_Loud (Hilbert envelope, 3-window stability) ===
\renewcommand{\NLIGOOK}{4113}
\renewcommand{\LIGOStableDelayed}{1558}
\renewcommand{\LIGOStableDelayedPct}{37.9}
\renewcommand{\LIGOStableFast}{1201}
\renewcommand{\LIGOStableFastPct}{29.2}
\renewcommand{\LIGOFlip}{1261}
\renewcommand{\LIGOFlipPct}{30.7}
\renewcommand{\LIGOFailed}{93}
\renewcommand{\NLIGOStable}{2759}
\renewcommand{\LIGOFailedPct}{2.3}
\renewcommand{\LIGOCurvatureAUC}{0.950}
\renewcommand{\LIGOCurvatureAUClo}{0.941}
\renewcommand{\LIGOCurvatureAUChi}{0.959}
\renewcommand{\LIGOCliffsDelta}{0.900}
% Key: curvature (b) separates strongly, survives SNR control; large flip fraction

% === LIGO Curvature Statistics (scaled to 10^-3) ===
\renewcommand{\LIGOCurvDelayedMedian}{0.06}
\renewcommand{\LIGOCurvDelayedIQRLo}{-0.4}
\renewcommand{\LIGOCurvDelayedIQRHi}{0.3}
\renewcommand{\LIGOCurvFastMedian}{-2.29}
\renewcommand{\LIGOCurvFastIQRLo}{-2.8}
\renewcommand{\LIGOCurvFastIQRHi}{-1.7}
% Mann-Whitney p-value exponent for curvature separation (from bootstrap_beta_b.json)
\renewcommand{\LIGOMWCurvPExp}{300}  % p < 10^-X

% === LIGO Bootstrap regression ===
% b-only model (Delayed ~ b)
\renewcommand{\LIGOBetaBOnly}{3.01}
\renewcommand{\LIGOBetaBOnlyLo}{2.77}
\renewcommand{\LIGOBetaBOnlyHi}{3.28}
\renewcommand{\LIGODeltaBICBOnly}{-2186.7}
\renewcommand{\LIGODeltaBICBOnlyLo}{-2333.9}
\renewcommand{\LIGODeltaBICBOnlyHi}{-2046.6}
% Full model (Delayed ~ b + SNR)
\renewcommand{\LIGOBetaB}{2.98}
\renewcommand{\LIGOBetaBLo}{2.75}
\renewcommand{\LIGOBetaBHi}{3.25}
\renewcommand{\LIGOBetaSNR}{-0.27}
\renewcommand{\LIGODeltaBIC}{-2145.1}
\renewcommand{\LIGODeltaBICLo}{-2286.5}
\renewcommand{\LIGODeltaBICHi}{-2006.9}
% SNR-only model (Delayed ~ SNR)
\renewcommand{\LIGOBetaSNROnly}{-0.29}

% === LIGO Threshold Sensitivity (T=1, T=4) ===
% T=1 (weaker evidence threshold)
\renewcommand{\LIGOThreshOneDelayedN}{1559}
\renewcommand{\LIGOThreshOneDelayedPct}{38.8}
\renewcommand{\LIGOThreshOneFastN}{466}
\renewcommand{\LIGOThreshOneFastPct}{11.6}
\renewcommand{\LIGOThreshOneFlipN}{1995}
\renewcommand{\LIGOThreshOneFlipPct}{49.6}
\renewcommand{\LIGOThreshOneAUC}{0.975}
% T=4 (stronger evidence threshold)
\renewcommand{\LIGOThreshFourDelayedN}{1503}
\renewcommand{\LIGOThreshFourDelayedPct}{37.4}
\renewcommand{\LIGOThreshFourFastN}{464}
\renewcommand{\LIGOThreshFourFastPct}{11.5}
\renewcommand{\LIGOThreshFourFlipN}{2053}
\renewcommand{\LIGOThreshFourFlipPct}{51.1}
\renewcommand{\LIGOThreshFourAUC}{0.977}

% === LIGO Flip Breakdown (3-window stability, consistent with LIGOFlip) ===
\renewcommand{\LIGOFlipDeterminateN}{137}
\renewcommand{\LIGOFlipDeterminatePct}{10.9}
\renewcommand{\LIGOFlipUncertainN}{1124}
\renewcommand{\LIGOFlipUncertainPct}{89.1}
% Flip direction: among uncertain flips, % with delayed geometry at each window
\renewcommand{\LIGOFlipShortDelayedPct}{67}  % Delayed at 60ms window
\renewcommand{\LIGOFlipLongDelayedPct}{33}  % Delayed at 150ms window

% === LIGO Hartigan's dip test ===
\renewcommand{\LIGODip}{0.024}
\renewcommand{\LIGODipP}{\ensuremath{<10^{-50}}}
\renewcommand{\LIGODipPDelayed}{0.99}
\renewcommand{\LIGODipPFast}{0.92}

% === LIGO Robustness: Dip test on ALL OK events (ignoring labels) ===
\renewcommand{\LIGODipUnlabeledN}{4020}
\renewcommand{\LIGODipUnlabeled}{0.004}
\renewcommand{\LIGODipUnlabeledP}{0.98}
\renewcommand{\LIGODipUnlabeledWinsP}{0.75}
\renewcommand{\LIGODipUnlabeledTrimP}{0.98}
\renewcommand{\LIGODipStableN}{2759}
\renewcommand{\LIGODipStableP}{\ensuremath{<10^{-50}}}
\renewcommand{\LIGODipStableTrimP}{\ensuremath{<10^{-50}}}

% === LIGO Robustness: Window-level regression with cluster-robust SEs ===
\renewcommand{\LIGOWindowN}{797}
\renewcommand{\LIGOWindowClusters}{272}
\renewcommand{\LIGOWindowBetaB}{1.61}
\renewcommand{\LIGOWindowZB}{5.02}
\renewcommand{\LIGOWindowPB}{0.000001}
\renewcommand{\LIGOWindowBetaSNR}{-0.31}
\renewcommand{\LIGOWindowPSNR}{0.0053}

% === LIGO Robustness: Unsupervised GMM validation ===
\renewcommand{\LIGOGMMAgreement}{88.9}
\renewcommand{\LIGOGMMARI}{0.60}
\renewcommand{\LIGOGMMNMI}{0.50}
\renewcommand{\LIGOGMMBoundary}{-1.13}

% === LIGO Robustness: Rejected-morphology stress test ===
\renewcommand{\LIGORejectN}{1941}
\renewcommand{\LIGORejectKS}{0.13}
\renewcommand{\LIGORejectKSP}{\ensuremath{4.2 \times 10^{-20}}}
\renewcommand{\LIGORejectDelayedN}{895}
\renewcommand{\LIGORejectFastN}{958}
\renewcommand{\LIGORejectUncertainN}{88}
\renewcommand{\LIGORejectDeterminateN}{1853}
\renewcommand{\LIGORejectAUC}{0.798}

% === LIGO Robustness: Alternative likelihood (log-domain) ===
\renewcommand{\LIGOAltLikelihoodAgreement}{65.3}

% === LIGO status ===
\providecommand{\LIGOStatus}{UNKNOWN}
\providecommand{\LIGOAsOf}{UNKNOWN}
\providecommand{\LIGOScaleNote}{}
\renewcommand{\LIGOStatus}{full catalog}
\renewcommand{\LIGOAsOf}{December 2025}
\renewcommand{\LIGOScaleNote}{The large flip fraction (30.7\%) indicates a substantial boundary population; stable-core discrimination remains strong (AUC=0.950).}

% === LIGO Curvature Window Sweep (robustness to fit interval choice) ===
% From ligo_curvature_sweep.py output
% 0-10ms window
\renewcommand{\LIGOCurvSweepTenAUC}{0.915}
\renewcommand{\LIGOCurvSweepTenCILo}{0.903}
\renewcommand{\LIGOCurvSweepTenCIHi}{0.926}
\renewcommand{\LIGOCurvSweepTenSign}{Yes}
% 0-20ms window (default)
\renewcommand{\LIGOCurvSweepTwentyAUC}{0.950}
\renewcommand{\LIGOCurvSweepTwentyCILo}{0.941}
\renewcommand{\LIGOCurvSweepTwentyCIHi}{0.959}
\renewcommand{\LIGOCurvSweepTwentySign}{Yes}
% 0-30ms window
\renewcommand{\LIGOCurvSweepThirtyAUC}{0.946}
\renewcommand{\LIGOCurvSweepThirtyCILo}{0.937}
\renewcommand{\LIGOCurvSweepThirtyCIHi}{0.955}
\renewcommand{\LIGOCurvSweepThirtySign}{No}
% Cross-window Spearman correlations (stable-core)
\renewcommand{\LIGOCurvSpearmanTenTwenty}{0.82}
\renewcommand{\LIGOCurvSpearmanTwentyThirty}{0.94}

% === LIGO Null Simulation Control (window-decoupled) ===
% From ligo_null_simulation.py - referee-grade control test
% Null-fast-only (pure fast-geometry world)
\renewcommand{\LIGONullN}{1000}
\renewcommand{\LIGONullMislabelRate}{1.50}
\renewcommand{\LIGONullMislabelCILo}{0.91}
\renewcommand{\LIGONullMislabelCIHi}{2.46}
\renewcommand{\LIGONullMislabeledN}{15}
\renewcommand{\LIGONullAmbiguousN}{244}
\renewcommand{\LIGONullAUCAssigned}{0.924}
\renewcommand{\LIGONullAUCRandom}{0.499}
% Mixed-truth (50% fast + 50% delayed with known ground truth)
\renewcommand{\LIGOMixedNTotal}{1000}
\renewcommand{\LIGOMixedNValid}{814}
\renewcommand{\LIGOMixedNAmbiguous}{186}
\renewcommand{\LIGOMixedAccuracy}{98.6}
\renewcommand{\LIGOMixedAUCTrue}{0.992}
\renewcommand{\LIGOMixedAUCAssigned}{0.991}
% Confusion matrix (true_pred counts)
\renewcommand{\LIGOMixedCMFF}{375}
\renewcommand{\LIGOMixedCMFD}{11}
\renewcommand{\LIGOMixedCMDF}{0}
\renewcommand{\LIGOMixedCMDD}{428}
% Null simulation configuration
\renewcommand{\LIGONullLateWindow}{[30, 100]}
\renewcommand{\LIGONullEarlyWindow}{[0, 20]}
% Parameter sweep (worst-case tuning)
\renewcommand{\LIGOSweepFixedMax}{2.67}
\renewcommand{\LIGOSweepFixedMaxCIHi}{5.17}
\renewcommand{\LIGOSweepWindowMax}{12.33}
\renewcommand{\LIGOSweepWindowMaxCIHi}{16.54}
\renewcommand{\LIGOSweepWorstWindow}{40--120}
\renewcommand{\LIGOSweepOverallMax}{12.33}
\renewcommand{\LIGOSweepOverallMaxCIHi}{16.54}
% Spliced-null control (cross-trace shuffle)
\renewcommand{\LIGOSplicedAUCUnspliced}{0.968}
\renewcommand{\LIGOSplicedAUCTrue}{0.989}
\renewcommand{\LIGOSplicedAUCMean}{0.500}
\renewcommand{\LIGOSplicedAUCStd}{0.019}
\renewcommand{\LIGOSplicedAUCDrop}{0.468}

% === Li et al. 63-qubit (baseline) ===
% Qualitative baseline only; no stable-core percentages reported due to pipeline differences
% All show fast exponential recovery (capacity-wins baseline)

% ===== END INLINED: results_macros.tex =====


% ============================================================
% DOCUMENT INFO
% ============================================================
\title{\textbf{Forensic Signatures Consistent with Bandwidth-Limited Recovery Dynamics in Qubits and LIGO Glitches}}
\author{Aernoud Dekker}
\date{December 2025 \\ \small Version 1.0}

% ============================================================
% DOCUMENT
% ============================================================
\begin{document}

\maketitle

\vspace{-1em}
\begin{center}
\small DOI: \href{https://doi.org/10.17605/OSF.IO/M5TB9}{10.17605/OSF.IO/M5TB9}
\end{center}
\vspace{1em}

% ------------------------------------------------------------
% ABSTRACT
% ------------------------------------------------------------
\begin{abstract}
We analyze recovery transients in precision control telemetry from superconducting quantum processors (26-qubit, 63-qubit) and LIGO gravitational-wave interferometry. We find a reproducible two-regime recovery geometry---fast (immediate-onset) vs.\ delayed (hesitation-like)---that persists under model, window, and seed-sweep robustness checks. A non-definitional curvature metric separates regimes with high discrimination (AUC $\approx \LIGOCurvatureAUC$), and unsupervised clustering aligns with the same partition (agreement $\approx \LIGOGMMAgreement$\%). Stable-core populations cluster distinctly with a boundary/flip subset; hesitation expression is platform-dependent (delay-dominant in 26-qubit, curvature-dominant in LIGO).

This is a retrospective pattern-finding analysis; causal attribution requires controlled bandwidth manipulation. We interpret the population structure as consistent with capacity-limited control near a threshold, as predicted by the bandwidth-limited quantum control (BLQC) framework \cite{dekker2025blqc}. The key prospective test: increasing effective controller bandwidth should shift events from delayed to fast geometry with a threshold-like transition; if geometry is purely passive physics, this dependence should be weak or absent.

\noindent\textit{Reproducibility: \ResultsBanner}
\end{abstract}

\newpage
\tableofcontents
\newpage

% ------------------------------------------------------------
% SECTIONS
% ------------------------------------------------------------
% ===== BEGIN INLINED: introduction.tex =====
% ============================================================
% SECTION 1: INTRODUCTION
% ============================================================
\section{Introduction}
\label{sec:introduction}

When precision instruments experience transient perturbations---cosmic ray impacts on superconducting qubits, glitches in gravitational wave interferometers---how do they recover? This paper documents geometric population structure in recovery dynamics across two precision-instrument platforms (26-qubit superconducting processor and LIGO gravitational-wave interferometry), with a third (63-qubit processor) providing a fast-recovery baseline.

The central empirical finding is within-platform heterogeneity: events on the same platform exhibit qualitatively distinct recovery geometries---some with immediate onset, others with delayed (hesitation-like) dynamics. This heterogeneity is not a continuous spectrum; events cluster into distinct populations with a boundary subset, suggesting two-regime dynamics. Strikingly, the same two-regime structure appears across platforms as different as superconducting qubits and km-scale interferometers.

A companion paper \cite{dekker2025blqc} presents a control-theoretic framework (bandwidth-limited quantum control, BLQC) that predicts this two-regime structure: when a controller's effective bandwidth is exceeded by internal dynamics, recovery shows delayed onset rather than immediate decay. The present paper is forensic: we document whether existing data show the predicted population structure, deferring causal attribution to controlled experiments.

Section~\ref{sec:protocol} describes the classification methodology. Sections~\ref{sec:chinese}--\ref{sec:ligo} present three case studies. Section~\ref{sec:cross} compares results across systems. Section~\ref{sec:discussion} discusses limitations and alternative explanations. Section~\ref{sec:conclusion} summarizes findings.

\smallskip
\noindent\textit{Analysis provenance: \ResultsBanner.}

% ===== END INLINED: introduction.tex =====

% ===== BEGIN INLINED: forensic_protocol.tex =====
% ============================================================
% SECTION 2: FORENSIC PROTOCOL
% ============================================================
\section{Forensic Protocol}
\label{sec:protocol}

\subsection{Observable Signature: Recovery Geometry}

The key experimental signature is the \textit{geometry} of recovery curves following perturbations. We analyze a normalized return-to-baseline variable $z(t)$ so that ``fast'' means maximum slope at $t = 0$ and ``delayed'' means maximum slope at $t > 0$, independent of whether the raw instrument signal rises or falls.

\begin{center}
\begin{tabular}{lcc}
\toprule
\textbf{Recovery Type} & \textbf{Model Geometry} & \textbf{Onset Time $t_{\mathrm{inf}}$} \\
\midrule
Immediate recovery & Fast (exponential-like) & $\approx 0$ \\
Delayed recovery & Hesitation (sigmoidal-like) & $> 0$ \\
\bottomrule
\end{tabular}
\end{center}

\textbf{Terminology:} ``Fast'' and ``Delayed'' refer to \textit{onset geometry}, not recovery duration. Fast geometry = immediate onset (max slope at $t=0$). Delayed geometry = onset after a plateau (max slope at $t>0$). Both may have similar overall timescales $\tau$; the distinction is the shape of onset, not the speed of completion.

\textbf{Platform-dependent expression:} Hesitation can manifest in two modes: (1) \textit{delay}---a temporal plateau before steep recovery, captured by onset delay fraction $D = t_{\mathrm{inf}}/\tau$; or (2) \textit{curvature}---flat-then-accelerating early-time geometry, captured by curvature index $b$ measured over a fixed early-time interval. Different platforms may express hesitation predominantly in one mode or the other, depending on sampling resolution and noise characteristics.

\subsection{Methodological Framework}

This paper presents a \textit{retrospective forensic analysis} of existing experimental datasets:
\begin{itemize}
\item \textbf{Exploratory, not confirmatory:} These analyses were not preregistered. They establish plausibility, not proof.
\item \textbf{Public data:} All datasets are publicly available (Appendix~\ref{app:scripts}; catalogs in Appendix~\ref{app:catalogs}).
\item \textbf{Model-based classification:} We use a parametric model tournament (AICc selection) rather than derivative-based methods.
\item \textbf{Stability analysis:} We classify across multiple windows and report stable-core populations separately from boundary/flip events.
\end{itemize}

\subsection{Scope and Sampling}
\label{sec:scope}

\begin{center}
\fbox{\parbox{0.9\textwidth}{
\textbf{Sampling disclosure:} LIGO results are based on the full deduplicated O3a Extremely Loud catalog (\LIGODedupUnique\ unique events from \LIGOAfterCuts\ after GPS deduplication). 26-qubit results use all available events from McEwen et al.\ (2022). Percentages reported throughout are \textit{conditional on analyzable single-pulse morphologies}---not prevalence estimates for the full populations.
}}
\end{center}

\begin{center}
\fbox{\parbox{0.9\textwidth}{
\textbf{Definitions and Notation}
\begin{itemize}
\item \textbf{$t = 0$}: Event extremum (envelope peak for LIGO; count minimum for qubit data)
\item \textbf{$t_{\mathrm{inf}}$}: Model-implied time of maximal recovery slope (inflection point); $t_{\mathrm{inf}} = 0$ for fast models, $t_{\mathrm{inf}} > 0$ for delayed models
\item \textbf{$D = t_{\mathrm{inf}}/\tau$}: Dimensionless delay fraction (by construction, $D > 0$ for delayed-geometry events)
\item \textbf{$b$}: Early-time curvature index from quadratic fit over 0--20~ms; $b > 0$ = accelerating recovery, $b < 0$ = decelerating
\item \textbf{Stable}: Same classification (fast/delayed) across all 3 analysis windows (60/100/150~ms)
\item \textbf{Flip}: Mixed classification across windows (boundary population)
\end{itemize}
}}
\end{center}

\subsection{The Classification Problem}

When analyzing recovery dynamics after perturbation events, we face a classification problem: does a given recovery curve represent \textit{immediate} recovery (capacity-wins) or \textit{delayed} recovery with hesitation (chaos-wins)?

Derivative-based approaches (finding $\arg\max |dy/dt|$) are sensitive to smoothing bandwidth, making them unreliable classifiers for discrete, noisy data. We therefore adopt a \textbf{model-based classification} approach that is substantially less sensitive to smoothing than derivative timing, with robustness assessed explicitly via multi-window stability analysis.

\subsection{Model Tournament with AICc Selection}

\subsubsection{Competing Models}

For each recovery event, we fit five parametric models representing two geometric classes:

\paragraph{Fast geometry (immediate recovery):}
\begin{itemize}
\item \textbf{Exponential:} $y(t) = y_\infty - A e^{-t/\tau}$
\item \textbf{Exponential (fixed baseline):} Same, with $y_\infty$ fixed to a measured asymptote (median over a terminal segment of the analysis window; see platform sections for details)
\item \textbf{Rational:} $y(t) = y_\infty - A/(1 + t/\tau)$
\end{itemize}

These models have maximum slope at $t = 0$; recovery begins immediately.

\paragraph{Delayed geometry (hesitation):}
\begin{itemize}
\item \textbf{Sigmoid:} $y(t) = y_{\min} + A/(1 + e^{-k(t - t_0)})$, with inflection at $t_0 > 0$; we define $\tau \equiv 1/k$ for comparability across models
\item \textbf{Delayed exponential:} $y(t) = y_{\min}$ for $t < t_d$, then exponential recovery with time constant $\tau$
\end{itemize}

These models have maximum slope at $t > 0$. Hesitation means the time of maximal slope occurs after $t = 0$, rather than immediately at the event onset.

\subsubsection{Model Selection}

We select the winning model using the corrected Akaike Information Criterion:
\begin{equation}
\text{AICc} = \text{AIC} + \frac{2k(k+1)}{n - k - 1}
\end{equation}
where $k$ is the number of parameters and $n$ is the number of data points. AICc penalizes overfitting more strongly than AIC for small samples \cite{burnham2002}.

\textbf{Statistical note:} AICc is computed from least-squares fits under an effective Gaussian residual approximation. We use it as a \textit{relative} model selection heuristic for geometry class, not as an absolute goodness-of-fit measure. 26-qubit (McEwen et al.) data are integer counts (Poisson-ish, heteroskedastic) and LIGO envelopes are positive amplitudes with non-Gaussian noise; neither satisfies Gaussian assumptions exactly. For coarse geometry discrimination among a restricted dictionary, rank-order is empirically stable under our robustness checks: replacing AICc with BIC yields the same stable-core split (identical classifications for all events), and the 3-window stability criterion (Section~\ref{sec:stability}) confirms classification robustness across analysis variations.

\subsubsection{Classification Rule}

Let $\text{AICc}_{\text{fast}}^{\min}$ be the minimum AICc among fast-geometry models and $\text{AICc}_{\text{del}}^{\min}$ the minimum among delayed-geometry models. Define the classwise gap:
\begin{equation}
\Delta\text{AICc} = \text{AICc}_{\text{fast}}^{\min} - \text{AICc}_{\text{del}}^{\min}
\end{equation}

Events are classified as:
\begin{itemize}
\item \textbf{Delayed (hesitation):} $\Delta\text{AICc} \geq 2$ (delayed geometry preferred)
\item \textbf{Fast (immediate):} $\Delta\text{AICc} \leq -2$ (fast geometry preferred)
\item \textbf{Uncertain:} $|\Delta\text{AICc}| < 2$ (honest acknowledgment of ambiguity)
\end{itemize}

The threshold $|\Delta\text{AICc}| \geq 2$ corresponds to conventional ``positive evidence'' in model comparison \cite{burnham2002}. (Robustness check: Appendix~\ref{app:robustness} reports results with alternative thresholds 1 and 4; stable-core identification is insensitive to this choice.)

\subsubsection{Derived Hesitation Coordinate}

From the winning model parameters, we compute the dimensionless delay fraction:
\begin{equation}
D = \frac{t_{\mathrm{inf}}}{\tau}
\end{equation}

\textbf{Definition of $t_{\mathrm{inf}}$:} The quantity $t_{\mathrm{inf}}$ is the \textit{model-implied} time of maximal recovery slope (inflection point), computed analytically from the fitted model parameters. \textbf{This is not the same as the signal peak time used for alignment.} Alignment is always to the signal extremum (peak or minimum); $t_{\mathrm{inf}}$ describes when the fitted model predicts the steepest recovery relative to that alignment.

Analytically: $t_{\mathrm{inf}} = 0$ for fast models (max slope at origin), $t_{\mathrm{inf}} = t_0$ for sigmoid (inflection parameter), $t_{\mathrm{inf}} = t_d$ for delayed exponential (delay parameter). Here $\tau$ is the recovery timescale: for delayed-exponential events, $\tau$ denotes the post-delay exponential time constant; for sigmoid events, $\tau \equiv 1/k$.

\textbf{Important:} Because delayed-geometry classification requires a delayed model to win, $D > 0$ for delayed-geometry events \textit{by construction}. Separation on $D$ is therefore expected given the classification rule and does not constitute non-definitional evidence. The nontrivial test is whether \textit{other} geometric features (curvature, SNR control) align with the classification.

\subsubsection{Model Selection Philosophy}
\label{sec:model-philosophy}

We intentionally use a restricted model dictionary as a \textit{coarse geometry probe}. The winning model serves as a label for shape class (fast/exponential-like vs.\ delayed/sigmoidal-like), not a claim about microphysics. More flexible families---stretched exponential, double exponential, ringdown-plus-recovery---could be added in future work; the core finding is the two-regime population separation, which survives all robustness tests reported in Appendix~\ref{app:robustness}. The model dictionary is sufficient to distinguish geometric classes; it is not intended to identify microscopic mechanisms.

\subsection{Early-Time Curvature Index}

As a model-independent geometric invariant, we compute the early-time curvature from the normalized recovery signal over a fixed window (0--20 ms post-extremum; the extremum is the envelope peak for LIGO, count minimum for qubit data). The 20 ms window was chosen as a pre-specified compromise: it spans approximately $2\tau$ for 26-qubit data ($\tau \sim 11$ ms) and captures the initial recovery onset in LIGO, while remaining short enough to probe early-time geometry rather than asymptotic behavior.

\subsubsection{Definition}

Normalize the signal to $z(t) \in [0, 1]$ where $z(0) \approx 0$ at the event extremum (peak or minimum) and $z \to 1$ as recovery completes. We work in this recovery coordinate so that $z(t)$ increases from the event extremum toward baseline, regardless of whether the raw instrument observable rises or falls. Fit a quadratic over the early window using $t$ in milliseconds:
\begin{equation}
z(t_{\mathrm{ms}}) \approx z_0 + a\,t_{\mathrm{ms}} + b\,t_{\mathrm{ms}}^2 \qquad (t_{\mathrm{ms}} \in [0, 20])
\label{eq:curvature}
\end{equation}
where $b$ has units $\mathrm{ms}^{-2}$. For readability, figures and tables report $10^3 b$ ($\mathrm{ms}^{-2}$); to recover $b$ in $\mathrm{ms}^{-2}$, divide the reported value by $10^3$.

The coefficient $b$ is the \textbf{curvature index}:
\begin{itemize}
\item $b > 0$: Slope increases with time (flat then accelerating recovery)---hesitation-like
\item $b < 0$: Slope decreases with time (steep then decelerating recovery)---standard
\item $b \approx 0$: Linear early recovery (boundary case)
\end{itemize}

\subsubsection{Sign Convention}

Our convention: $z(t)$ increases from 0 toward 1 during recovery. A positive $b$ means recovery \textit{accelerates} (slope increases over time), producing flat-then-steep geometry characteristic of hesitation. A negative $b$ means recovery \textit{decelerates} (slope decreases), producing steep-then-flat geometry characteristic of immediate recovery.

\subsubsection{Why Curvature Matters}

Unlike $D$, the curvature index $b$ is computed directly from the time series, not from the model tournament. It measures the \textit{shape} of early recovery. If $b$ separates delayed from fast events, this provides classifier-external geometric consistency---a shape statistic that aligns with the model-derived labels without being definitionally tied to them. It is computed from the same time series (so not statistically independent), but it is not definitionally tied to the tournament: it is a fixed-window quadratic shape statistic rather than a fitted recovery model. (Sensitivity to fit-interval choice and baseline method is reported in Appendix~\ref{app:curvature-window} and~\ref{app:baseline}.)

\subsubsection{Time-Origin Consistency}

Events are aligned to the dataset-native trigger (e.g., glitch peak time for LIGO; event timestamp/onset marker for qubit telemetry); the geometry criteria are defined relative to that alignment and are not assumed to share a universal absolute $t=0$ across platforms.

\subsection{Stability Analysis}
\label{sec:stability}

To ensure classifications are robust to methodological choices, we analyze each event at multiple window sizes (60, 100, 150 ms) with consistent alignment. An event is classified as:
\begin{itemize}
\item \textbf{Stable Delayed:} Delayed classification in all 3/3 windows
\item \textbf{Stable Fast:} Fast classification in all 3/3 windows
\item \textbf{Flip:} Any mixed result across windows, subdivided into:
  \begin{itemize}
  \item \textit{Flip-determinate:} Fast vs.\ Delayed disagreement across windows (genuine boundary events)
  \item \textit{Flip-uncertain:} At least one window classified as Uncertain (ambiguous fit quality)
  \end{itemize}
\end{itemize}

This subdivision separates genuine boundary events (where the classification changes determinately) from cases where fit quality is ambiguous. The flip rate quantifies the fraction of events near the decision boundary or with unstable fits. Distinct stable-core populations with a moderate flip fraction indicate genuine two-population structure; an extremely high flip rate (approaching 50\%) with no stable cores would suggest arbitrary classification. (Flip subtypes, $\Delta$AICc gap distributions, and threshold sensitivity are detailed in Appendix~\ref{app:flip} and~\ref{app:threshold}.)

\subsection{Selection Funnel}

Not all events in a dataset are suitable for recovery dynamics analysis. We apply explicit selection criteria and report the full funnel:

\begin{enumerate}
\item \textbf{Population:} Define the source population (e.g., all glitches of a given class with SNR $\geq$ threshold)
\item \textbf{Data retrieval:} Note any subset limitations (e.g., strain data availability)
\item \textbf{Morphology checks:} Apply sanity filters appropriate to the platform:
\begin{itemize}
\item \textit{Peak dominance:} Reject multi-component events (secondary peaks $> 70\%$ of primary)
\item \textit{Decay trend:} Require overall decay post-peak (Theil-Sen slope $< 0$ on log-signal)
\end{itemize}
\item \textbf{Analyzable:} Events passing all checks proceed to model fitting
\end{enumerate}

Results are reported as conditional on passing these gates. This transparency prevents accusations of selection bias while acknowledging that conclusions apply to the analyzable subset, not necessarily to all events. Nothing in the funnel is tuned to produce a particular outcome; the morphology checks are designed for analyzable single-pulse events, not for maximizing delayed classifications. We are not estimating prevalence in the full glitch population; we are characterizing geometry conditional on analyzable single-pulse morphology.

\subsection{Confound Control}

A potential concern is that the hesitation signature might simply reflect event magnitude: louder/deeper events might ``look different'' for mundane reasons. We address this with logistic regression:
\begin{equation}
\text{logit}(P(\text{Delayed})) = \beta_0 + \beta_b \cdot b_{\text{std}} + \beta_{\text{SNR}} \cdot \text{SNR}_{\text{std}}
\end{equation}
where predictors are standardized (zero mean, unit variance).

Key tests:
\begin{itemize}
\item Does $\beta_b$ remain significant after controlling for SNR?
\item Is $\beta_{\text{SNR}}$ positive (loud $\Rightarrow$ delayed) or negative/null?
\item Does the full model outperform SNR-only by $\Delta$BIC?
\end{itemize}

Bootstrap resampling (1000 iterations) provides 95\% confidence intervals for all coefficients, ensuring results are not overfit artifacts.

\subsection{Summary of Protocol}

\begin{center}
\begin{tabular}{ll}
\toprule
\textbf{Component} & \textbf{Method} \\
\midrule
Classification & AICc model tournament (5 models, 2 geometries) \\
Primary feature & Winning model geometry (fast vs.\ delayed) \\
Derived coordinate & $D = t_{\mathrm{inf}}/\tau$ (by construction for delayed) \\
Independent geometry & Curvature index $b$ from Eq.~\ref{eq:curvature} \\
Robustness & Stability across 60/100/150 ms windows \\
Confound control & Logistic regression with SNR, bootstrap CIs \\
Transparency & Explicit selection funnel per platform \\
\bottomrule
\end{tabular}
\end{center}

The model family and selection rule are identical across platforms; only the construction of the recovery coordinate $z(t)$ differs (e.g., Hilbert envelope for LIGO, count inversion for qubit telemetry). Platform-specific preprocessing is documented in the relevant sections.

% ===== END INLINED: forensic_protocol.tex =====

% ===== BEGIN INLINED: chinese_cosmic_ray.tex =====
% ============================================================
% SECTION 3: 63-QUBIT COSMIC RAY STUDY (Li et al. 2025)
% ============================================================
\section{Case Study I: Capacity-Wins Regime (63-Qubit Processor)}
\label{sec:chinese}

\noindent\textit{Note: This case study provides illustrative contrast only---not part of cross-platform percentage comparisons. Due to different observables and pipeline constraints (limited high-resolution event counts, no 3-window stability protocol), we do not include these data in the quantitative summary tables. The section demonstrates what ``capacity-wins'' recovery looks like as a qualitative baseline.}

\subsection{Experimental Context}

Li et al.\ (2025) \cite{li2025} published a comprehensive study of cosmic-ray-induced correlated errors in a 63-qubit superconducting processor. The experiment featured several innovations relevant to our analysis:

\begin{itemize}
\item \textbf{Large qubit array:} 63 superconducting transmon qubits, providing statistical power
\item \textbf{High time resolution:} Charge-parity jump monitoring at 5.6~$\mu$s effective sampling
\item \textbf{Direct cosmic ray correlation:} Muon detectors integrated into the dilution refrigerator
\item \textbf{Separated contributions:} Cosmic-ray muons distinguished from background $\gamma$-rays
\end{itemize}

\subsection{Data Description}

The publicly available dataset (Figshare, DOI: 10.6084/m9.figshare.28815434) includes:

\begin{center}
\begin{tabular}{lcc}
\toprule
\textbf{Measurement} & \textbf{Resolution} & \textbf{Duration} \\
\midrule
Charge-parity jumps (P\_MQSCPJ) & 5.6 $\mu$s & 188 seconds \\
Bit flip probability (P\_MQSBF) & 56 ms & 2 hours \\
\bottomrule
\end{tabular}
\end{center}

The high-resolution charge-parity data comprises 33.5 million data points, providing exceptional temporal detail for recovery dynamics analysis.

\subsection{Results: Fast Exponential Recovery}

We identified 10 burst events in the high-resolution charge-parity data and 82 events in the bit flip monitoring data. The key finding is \textit{fast recovery}:

\begin{center}
\begin{tabular}{lcc}
\toprule
\textbf{Observable} & \textbf{Recovery to 50\%} & \textbf{Classification} \\
\midrule
Charge-parity jumps (N=10) & 22--213 $\mu$s & Standard (fast-geometry) \\
Bit flip probability (N=82) & $\sim$0.3--2 s (order of magnitude) & Standard (fast-geometry) \\
\bottomrule
\end{tabular}
\end{center}

\begin{figure}[H]
\centering
\includegraphics[width=0.9\textwidth]{../figures/chinese/chinese_cosmic_ray_events.png}
\caption{Example cosmic ray events from the 63-qubit processor (Li et al., 2025) showing charge-parity jump dynamics. Recovery is fast and approximately exponential, with steepest slope at or near $t = 0$. This is consistent with ``capacity-wins'' operation where the effective monitoring bandwidth exceeds the perturbation rate.}
\label{fig:chinese_cpj}
\end{figure}

The charge-parity recovery times (sub-millisecond) are 10--100$\times$ faster than the 26-qubit error-count telemetry (10--100~ms), highlighting a large timescale separation across platforms and observables, consistent with very different effective $C/\lambda$ ratios.

\paragraph{Protocol note.}
The 63-qubit dataset provides limited event counts in the high-resolution channel (10 bursts) and uses different observables than Sections~\ref{sec:google}--\ref{sec:ligo}. We therefore treat it as a qualitative baseline: events show immediate-onset recovery with no visible plateau. Where model fitting is feasible, fast-geometry models dominate and $t_{\mathrm{inf}}\approx 0$.

\subsection{BLQC Interpretation: The Capacity-Wins Baseline}

The fast, exponential recovery observed in the 63-qubit data is consistent with what BLQC predicts in the \textbf{capacity-wins regime} ($C_{\rm eff} > \lambda / \ln 2$):

\begin{itemize}
\item The effective monitoring/control bandwidth ($C_{\rm eff}$) is high enough to track the perturbation
\item No ``tracking-loss phase'' is observed---the monitoring process does not enter a saturated regime
\item Recovery is immediate and monotonically decelerating
\item The steepest slope occurs at $t = 0$
\end{itemize}

We emphasize that in superconducting qubits the relevant ``observer'' may be a mixture of physical relaxation plus classical monitoring/control; we use $C$ and $\lambda$ as effective parameters describing the net recovery geometry, not as a claim about a specific feedback implementation.

\subsection{Comparison with BLQC Predictions}

\paragraph{Why capacity-wins here?}
At minimum, two differences from the Sycamore telemetry are relevant: (i) the observable (charge-parity jumps / bit-flip probability rather than aggregate cycle error counts), and (ii) the effective monitoring cadence (microseconds vs 100~$\mu$s). These differences plausibly shift the effective $C/\lambda$ ratio toward immediate-onset recovery.

BLQC does not predict that \textit{all} systems should show sigmoidal recovery. Rather, it predicts a transition between regimes based on the ratio $C/\lambda$. Systems with high $C$ (fast controllers) should show exponential recovery; systems with marginal $C$ should show sigmoidal recovery when pushed hard.

\subsection{Significance: A Contrast Case}

The 63-qubit data serves as a \textit{contrast case}, not an apples-to-apples prevalence comparison. It demonstrates that fast exponential recovery is achievable in superconducting qubit systems, establishing a baseline against which the 26-qubit and LIGO findings can be interpreted. We do not infer mechanism or prevalence across platforms from this comparison---only that two distinct recovery geometries exist across the combined body of evidence.

Specifically, the 63-qubit data provides:
\begin{itemize}
\item A demonstration of what ``immediate-onset'' (fast-geometry) recovery looks like
\item Evidence that fast exponential recovery \textit{is} achievable in quantum systems
\item A meaningful contrast with 26-qubit data (Section~\ref{sec:google}), where a delayed-geometry subpopulation is observed
\end{itemize}

If all cosmic ray recovery were sigmoidal, we might suspect an artifact of the measurement process. The fact that \textit{some} systems show clean exponential recovery while \textit{others} show sigmoidal dynamics is consistent with two-regime population structure---whether from BLQC's capacity threshold or other mechanisms.

\begin{figure}[H]
\centering
\includegraphics[width=0.9\textwidth]{../figures/chinese/chinese_bitflip_events.png}
\caption{Bit flip probability events from the 63-qubit processor (Li et al., 2025). While showing slower recovery than charge-parity jumps (consistent with different physical mechanisms), the dynamics remain exponential with steepest slope at early times.}
\label{fig:chinese_bf}
\end{figure}

% ===== END INLINED: chinese_cosmic_ray.tex =====

% ===== BEGIN INLINED: google_quantum_ai.tex =====
% ============================================================
% SECTION 4: 26-QUBIT PROCESSOR (MIXED POPULATION)
% ============================================================
\section{Case Study II: Mixed Population (26-Qubit Processor)}
\label{sec:google}

\subsection{Experimental Context}

McEwen et al.\ (2022) \cite{mcewen2022} published a landmark study on cosmic ray impacts in a 26-qubit Sycamore processor. The experiment documented catastrophic error bursts affecting multiple qubits simultaneously:

\begin{itemize}
\item \textbf{System:} 26 superconducting transmon qubits on Sycamore chip
\item \textbf{Operating temperature:} $\sim$15 millikelvin
\item \textbf{Time resolution:} 100 $\mu$s sampling interval
\item \textbf{Monitoring:} Total error count per measurement cycle
\end{itemize}

\paragraph{The Cosmic Ray Effect:} When a high-energy particle strikes the chip substrate:
\begin{enumerate}
\item Energy deposits into the silicon substrate
\item Cooper pairs break, creating a flood of quasiparticles
\item Quasiparticles cause rapid $T_1$ relaxation (qubits decay to $|0\rangle$)
\item Error count temporarily \textit{drops} (correlated relaxation suppresses stochastic error signatures)
\item As quasiparticles dissipate, normal error rates return
\end{enumerate}

We analyze the \textit{recovery phase}---the dynamics of returning to baseline error rates---to search for BLQC signatures.

\subsection{Dataset Description}

\begin{center}
\begin{tabular}{ll}
\toprule
\textbf{Parameter} & \textbf{Value} \\
\midrule
Files analyzed & 91 (MAIN\_DATASETS) \\
Duration per file & 60 seconds \\
Sampling interval & 100 $\mu$s \\
Points per file & 600,000 \\
Baseline error rate & $\sim$23 errors/cycle \\
\bottomrule
\end{tabular}
\end{center}

Data publicly available at Figshare (DOI: 10.6084/m9.figshare.16673851).

\subsection{Event Detection and Classification}

We detected cosmic ray events as drops exceeding 5$\sigma$ below baseline, requiring minimum 500~ms separation between events. Each event was classified using the model tournament described in Section~\ref{sec:protocol}, with stability assessed across multiple analysis windows.

\begin{center}
\begin{tabular}{lcc}
\toprule
\textbf{Category} & \textbf{N} & \textbf{Percentage} \\
\midrule
Total events detected & \NMcEwen & 100\% \\
Stable Fast (immediate recovery) & \McEwenStableFast & \McEwenStableFastPct\% \\
Stable Delayed (delayed onset) & \McEwenStableDelayed & \McEwenStableDelayedPct\% \\
Flip (boundary population) & \McEwenFlip & \McEwenFlipPct\% \\
\bottomrule
\end{tabular}
\end{center}

\textbf{Key finding:} A small but stable hesitation subpopulation (\McEwenStableDelayedPct\%; Wilson 95\% CI: [\McEwenStableDelayedCILo\%, \McEwenStableDelayedCIHi\%]) shows delayed recovery onset. The majority of events recover immediately, consistent with capacity-wins dynamics.

\subsection{Recovery Dynamics}

When events are grouped by their model-based stability classification (stable delayed vs stable fast), two distinct recovery profiles emerge. The stable delayed population shows a characteristic plateau before the onset of steep recovery, while the stable fast population shows immediate exponential-like recovery.

\subsubsection{The Stable Fast Population}

Events classified as stable fast (n=\McEwenStableFast, \McEwenStableFastPct\%) show:
\begin{itemize}
\item \textbf{Best model:} Exponential or rational (fast geometry wins AICc tournament)
\item \textbf{Onset delay:} $D \approx 0$ (recovery begins immediately)
\item \textbf{Time constant:} $\tau \approx \McEwenTauMedian$~ms (median; IQR \McEwenTauIQRLo--\McEwenTauIQRHi~ms)
\end{itemize}

This is the ``normal'' recovery expected from standard quasiparticle physics---the capacity-wins regime.

\subsubsection{The Stable Delayed Population}

Events classified as stable delayed (n=\McEwenStableDelayed, \McEwenStableDelayedPct\%) show:
\begin{itemize}
\item \textbf{Best model:} Sigmoid or delayed-onset (delayed geometry wins AICc tournament)
\item \textbf{Onset delay:} $D > 0$ (measurable plateau before steep recovery)
\item \textbf{Initial phase:} Flat region where recovery is suppressed
\end{itemize}

This is the ``tracking-loss phase'' predicted by BLQC when $\lambda > C \ln 2$. As discussed in Section~\ref{sec:cross}, 26-qubit data primarily exhibit \textit{delay mode} hesitation: the onset of steep recovery is delayed, but once recovery begins, the geometry is similar to STD events.

\subsection{Hypothesis for Future Testing}

\textit{We pre-specify the following hypothesis for future preregistered testing; it is not a finding of this paper.}

BLQC predicts that deeper perturbations are more likely to exceed the capacity threshold. With only \McEwenStableDelayed\ stable delayed events in this dataset, we lack statistical power to test this prediction. We therefore state the hypothesis---severity-dependence of classification---for formal preregistered testing in future work with larger samples or controlled perturbation depths. We make no claim about effect direction or magnitude from the present data.

\subsection{Delay Mode, Not Curvature Mode}

In the 26-qubit (McEwen et al.) platform, the distinguishing geometric feature is not a curvature deformation during recovery, but a \emph{temporal offset before recovery begins}. Delayed-geometry events exhibit a plateau (delay mode) followed by recovery with similar curvature to the standard population.

BLQC predicts two regimes:
\begin{itemize}
\item \textbf{Capacity-wins:} Recovery begins immediately at $t = 0$
\item \textbf{Chaos-wins:} Recovery is delayed by a ``tracking-loss phase'' (plateau/shoulder)
\end{itemize}

The 26-qubit data are consistent with this prediction: stable delayed events show nonzero onset delay ($D > 0$), while stable fast events begin recovering immediately ($D \approx 0$). Once recovery begins, both populations proceed with similar dynamics (see Section~\ref{sec:google_curvature} and Figure~\ref{fig:mcewan_curvature}).

\subsection{Note on Timescales (Speculative)}

BLQC proposes that hesitation may arise from a capacity--instability tradeoff, while the absolute timescale depends on system-specific parameters (controller bandwidth, sampling rate, feedback architecture). The observed recovery timescale in 26-qubit data ($\tau \sim \McEwenTauMedian$~ms) reflects the quasiparticle recombination dynamics and controller response, not a universal constant.

The present work does not test universality across disparate observer classes; we restrict our claims to the empirical existence of hesitation-like transients in the analyzed platforms. Whether the same information-theoretic mechanism underlies hesitation in biological, quantum, and classical control systems remains an open question for future investigation.

\subsection{Early-Time Curvature: Delay Mode Only}
\label{sec:google_curvature}

Unlike LIGO data (Section~\ref{sec:ligo}), the early-time curvature index $b$ does \textit{not} discriminate between delayed and fast events in the 26-qubit (McEwen et al.) data (Mann-Whitney $p = \McEwenCurvatureP$). As shown in Figure~\ref{fig:mcewan_curvature}, both populations have similar curvature distributions once recovery begins.

\begin{figure}[H]
\centering
\includegraphics[width=0.95\textwidth]{../figures/google/mcewan_curvature_index.png}
\caption{Early-time curvature in 26-qubit cosmic ray recovery (McEwen et al., 2022). \textbf{Left:} Boxplot showing overlapping curvature distributions for Stable Fast, Flip, and Stable Delayed events. Unlike LIGO, curvature does not discriminate between populations ($p = \McEwenCurvatureP$). \textbf{Right:} Density distributions confirming substantial overlap. In this platform, hesitation is encoded entirely in the delay parameter $D$ (time offset before recovery), not in the shape of recovery once it begins.}
\label{fig:mcewan_curvature}
\end{figure}

This confirms that 26-qubit data exhibit \textit{delay mode} hesitation: the onset of recovery is delayed, but once recovery starts, it proceeds with standard geometry. The hesitation signature is a temporal offset, not a shape deformation.

\subsection{Robustness: Evidence Strength Across Parameter Variations}
\label{sec:google_robustness}

To assess the stability of our classification under analysis choices, we swept 15 configurations varying window length (60, 100, 150~ms), alignment method (argmin vs threshold), baseline treatment (fixed vs free), and model set composition. This yields \McEwenRobustN\ event--configuration pairs (15 configurations $\times$ 230 classified events).

\textbf{Key finding:} Events in the stable delayed class are strongly enriched for moderate evidence: 60\% exceed $|\Delta\mathrm{AICc}| \ge 4$ versus 6\% of flip events (RR $\approx$ 9.8, OR $\approx$ 23, Fisher exact $p \sim 10^{-6}$). Notably, the absolute counts above threshold are the same (9 events) but occur in vastly different denominators (15 vs 147), indicating enrichment rather than a global shift.

Stable delayed events show median $|\Delta\mathrm{AICc}| = \McEwenStableIOFEvidMed$ (IQR: \McEwenStableIOFEvidIQRLo--\McEwenStableIOFEvidIQRHi), with 23\% (3--4/15) exceeding the strong/decisive threshold ($|\Delta\mathrm{AICc}| \ge 10$). Flip-class events have median $|\Delta\mathrm{AICc}| = \McEwenFlipEvidMed$ (IQR: \McEwenFlipEvidIQRLo--\McEwenFlipEvidIQRHi).

Because flip events overwhelmingly sit below the $|\Delta\mathrm{AICc}| = 4$ separation threshold, their parameter sensitivity is expected: small perturbations in alignment or windowing can legitimately swap the best model when models are near-tied. The stable delayed population, by contrast, is concentrated among events where one geometry genuinely dominates.

The three-tier structure therefore reflects real population heterogeneity rather than analysis sensitivity:
\begin{itemize}
\item \textbf{Stable delayed} (\McEwenStableDelayed\ events): Consistently prefer delayed geometry across all 15 configurations; moderate-to-strong evidence.
\item \textbf{Stable fast} (\McEwenStableFast\ events): Consistently prefer fast geometry; moderate evidence.
\item \textbf{Flip} (\McEwenFlip\ events): Classification depends on analysis choices; genuinely weak evidence ($|\Delta\mathrm{AICc}| < 4$).
\end{itemize}

Figure~\ref{fig:google_robustness} shows the evidence-strength distribution by class, confirming that flip events cluster in the low-evidence regime where model selection is inherently indeterminate.

\begin{figure}[H]
\centering
\includegraphics[width=0.95\textwidth]{../figures/google/stability_evidence_strength.png}
\caption{Evidence strength by stability class across the 15-configuration robustness sweep. \textbf{Left:} Boxplots showing median $|\Delta\mathrm{AICc}|$ for each class. Stable BLQC (delayed-geometry) events have consistently higher evidence than flip or stable STD events. \textbf{Right:} Density distributions showing the flip population concentrates in the low-evidence regime ($|\Delta\mathrm{AICc}| < 4$) where classification boundaries are genuinely ambiguous. Horizontal lines mark the ``strong'' ($\ge 4$) and ``very strong'' ($\ge 10$) evidence thresholds.}
\label{fig:google_robustness}
\end{figure}

\subsection{Summary of 26-Qubit Findings}

\begin{enumerate}
\item \textbf{Population structure:} \NMcEwen\ events $\rightarrow$ \McEwenStableFast\ stable fast (\McEwenStableFastPct\%), \McEwenStableDelayed\ stable delayed (\McEwenStableDelayedPct\%), \McEwenFlip\ boundary/flip (\McEwenFlipPct\%)
\item \textbf{Primary signature:} Delayed-geometry events show nonzero onset delay ($D > 0$); once recovery begins, curvature is similar across classes
\item \textbf{Limitation:} Curvature does not discriminate populations in this platform ($p = \McEwenCurvatureP$); the evidence rests on geometry classification alone
\item \textbf{Methodology note:} Model-based classification is substantially less sensitive to preprocessing choices than derivative-based approaches
\end{enumerate}

\textbf{Evidence status:} In the 26-qubit dataset, $D$ is not treated as non-definitional evidence because it is derived from the same model fit that defines the geometry class; we therefore use this dataset primarily to demonstrate the existence of two recovery geometries in raw traces, not to validate a specific mechanism. Non-definitional consistency comes from LIGO (curvature separation + label-external checks), not from 26-qubit data.

A small set of events (\McEwenStableDelayed/\NMcEwen) consistently prefers delayed geometry across analysis windows. This provides weak evidence for hesitation-like dynamics---sufficient to warrant prospective tests that directly manipulate effective controller bandwidth, but not strong enough (absent classifier-external discrimination) to claim a robust BLQC signature.

% ===== END INLINED: google_quantum_ai.tex =====

% ===== BEGIN INLINED: ligo_virgo.tex =====
% ============================================================
% SECTION: LIGO GLITCH ANALYSIS (UPDATED METHODOLOGY)
% ============================================================
\section{Case Study III: LIGO Gravitational Wave Detector}
\label{sec:ligo}

\subsection{Motivation}

The Advanced LIGO gravitational wave interferometers represent precision measurement systems operating at the edge of quantum and classical limits. Each detector comprises 4-km arm cavities with suspended mirrors, nested feedback loops for length and alignment control, and continuous operation near marginal stability.

The Gravity Spy citizen science project \cite{gravityspy} has cataloged and classified transient noise events (``glitches'') using machine learning, providing a large dataset of perturbation-recovery dynamics. BLQC predicts that signatures of controller effective bandwidth limits---specifically, delayed recovery or ``hesitation''---should appear when control systems are pushed beyond their tracking capacity.

\subsection{Data and Methods}

\subsubsection{Data Source}

We analyzed glitches from the LIGO Hanford (H1) detector during observing run O3a, accessed via the Gravitational Wave Open Science Center (GWOSC) \cite{gwosc}. From the Gravity Spy catalog:

\begin{center}
\begin{tabular}{lc}
\toprule
\textbf{Selection Criterion} & \textbf{N} \\
\midrule
Total H1 O3a glitches & \LIGOAllGlitches \\
``Extremely Loud'' class & \LIGOExtremelyLoudTotal \\
SNR $\geq$ 50, confidence $\geq$ 0.8 & \LIGOAfterCuts \\
\bottomrule
\end{tabular}
\end{center}

\textbf{Catalog preprocessing and sampling.} We sampled events from the Gravity Spy glitch catalog after deduplicating entries by (IFO, GPS time) to ensure each physical event was represented once. When multiple catalog rows referred to the same event, we retained the entry with the maximum SNR. All downstream counts and rates are computed per unique event. See Section~\ref{sec:scope} for sampling scope.

\subsubsection{Signal Extraction: Hilbert Envelope}

Previous analyses of LIGO glitch morphology often use the absolute value of strain, $|h(t)|$. However, this introduces zero-crossing artifacts that can obscure the underlying amplitude dynamics. We instead compute the \textbf{Hilbert envelope}:
\begin{equation}
E(t) = |h(t) + i\mathcal{H}[h(t)]|
\end{equation}
where $\mathcal{H}$ denotes the Hilbert transform. This provides the instantaneous amplitude without folding artifacts, appropriate for oscillatory transients.

Strain data is obtained at 4096~Hz from GWOSC open data (4~kHz files). Prior to envelope extraction, strain is bandpass filtered (10--500~Hz, 4th-order Butterworth) in zero-phase form (forward--backward) to avoid phase shifts while removing low-frequency drift and high-frequency noise.

\paragraph{Sample rate justification.}
All analyses use 4096~Hz open strain. Our features are envelope-based and integrated over 0--150~ms windows; 4096~Hz provides $\sim$0.244~ms resolution ($\sim$80 samples in the 0--20~ms curvature window). Nothing in our method depends on resolving sub-sample ($<$0.24~ms) structure; 16~kHz would increase storage/compute without changing the modeled timescales. Scripts support alternative sample rates for readers who wish to replicate at higher resolution.

\subsubsection{Sanity Checks}

Not all glitches are suitable for recovery dynamics analysis. We apply two sanity checks to select single-pulse events:

\begin{enumerate}
\item \textbf{Peak dominance:} The envelope must have a single dominant peak (secondary peaks $< 70\%$ of primary; heuristic single-pulse criterion). This rejects multi-component events.
\item \textbf{Decay trend:} The post-peak envelope must show overall decay (Theil-Sen slope $< -1\,\mathrm{s}^{-1}$ in $\ln E$, equivalently $< -0.001\,\mathrm{ms}^{-1}$, computed over $[0, 100]$ ms post-peak; threshold chosen to reject non-decaying envelopes, not tuned).
\end{enumerate}

Thresholds are fixed a priori for this analysis and reported explicitly. Events failing these checks are classified as ``complex'' and excluded from model fitting.

\subsubsection{Model-Based Classification}

For events passing sanity checks, we fit five recovery models to the normalized envelope $z(t) = 1 - (E(t) - E_\infty)/(E_{\text{peak}} - E_\infty)$ over a post-peak window. The peak time is $\arg\max E(t)$ within the fetched segment; windows are aligned to this peak. We estimate $E_\infty$ as the median envelope value over the final 10~ms of each analysis window. Events with $E_{\text{peak}}/E_\infty < 1.5$ are excluded to avoid near-singular normalization. Thus $z(t)$ starts near 0 at the glitch peak and rises toward 1 as the envelope relaxes to baseline.

\begin{center}
\begin{tabular}{lll}
\toprule
\textbf{Model} & \textbf{Geometry} & \textbf{Characteristic} \\
\midrule
Exponential & Fast & Immediate steep recovery \\
Exp.\ fixed baseline & Fast & Immediate steep recovery \\
Rational & Fast & Non-exponential fast recovery (heavy tail) \\
Sigmoid & Delayed & S-curve with inflection \\
Delayed exponential & Delayed & Plateau then recovery \\
\bottomrule
\end{tabular}
\end{center}

Model selection uses the corrected Akaike Information Criterion (AICc). With $\Delta\text{AICc} = \text{AICc}_{\text{fast}}^{\min} - \text{AICc}_{\text{del}}^{\min}$ (see Section~\ref{sec:protocol}), events are classified as:
\begin{itemize}
\item \textbf{Delayed (hesitation):} $\Delta\text{AICc} \geq 2$
\item \textbf{Fast (immediate):} $\Delta\text{AICc} \leq -2$
\item \textbf{Uncertain:} $|\Delta\text{AICc}| < 2$
\end{itemize}

\subsubsection{Stability Analysis}

To ensure classifications are robust to methodological choices, we analyze each event at three window sizes (60, 100, 150 ms) with peak alignment. Uncertain windows are treated as non-agreement; thus stability requires 3/3 determinate classifications of the same type:
\begin{itemize}
\item \textbf{Stable Delayed:} Delayed classification in all 3/3 windows
\item \textbf{Stable Fast:} Fast classification in all 3/3 windows
\item \textbf{Flip:} Any mixed or Uncertain result across windows
\end{itemize}

\subsubsection{Early-Time Curvature Index}

As a model-independent geometric invariant, we compute the early-time curvature from the normalized envelope over 0--20 ms post-peak:
\begin{equation}
z(t_{\mathrm{ms}}) \approx z_0 + a\,t_{\mathrm{ms}} + b\,t_{\mathrm{ms}}^2 \qquad (t_{\mathrm{ms}} \in [0, 20])
\end{equation}
Here $b$ has units ms$^{-2}$; we report $10^{3} b$ (ms$^{-2}$) in tables and figures. The coefficient $b$ (curvature index) captures the shape of the initial recovery:
\begin{itemize}
\item $b > 0$: Slope increases with time (flat then accelerating recovery)
\item $b < 0$: Slope decreases with time (steep then slowing recovery)
\end{itemize}
Delayed events show less-negative or upward-shifted curvature (often near-zero) relative to fast events; discrimination is quantified by AUC and effect size, not by a hard sign threshold.

\subsection{Selection Funnel}

We report the full selection pipeline to ensure transparency:

\begin{center}
\begin{tabular}{lrl}
\toprule
\textbf{Stage} & \textbf{N} & \textbf{Notes} \\
\midrule
Population (SNR$\geq$50, conf$\geq$0.8) & \LIGOAfterCuts & Full H1 O3a Extremely Loud \\
After GPS deduplication & \LIGODedupUnique & Unique physical events \\
Passed sanity + fetch succeeded & \NLIGOOK & Single-pulse morphologies analyzed \\
\quad $\hookrightarrow$ Fit-failed subset & \LIGOFailed & Excluded from curvature/regression \\
\bottomrule
\end{tabular}
\end{center}

\textbf{Critical framing:} Results apply to \textit{analyzable single-pulse morphologies} under our sanity criteria---not to all Extremely Loud glitches. The analyzed subset represents events for which strain data was successfully fetched and which passed morphology sanity checks (single dominant peak, overall decay trend). Events failing these checks typically exhibit multi-component structure or lack clear decay trends.

Here, \textbf{OK} denotes events that pass morphology sanity checks and proceed to fitting; a subset may fail numerical fitting in one or more windows during stability analysis and is reported separately as \textbf{Failed}. Failed events are excluded from curvature/regression analyses but included in stability tallies.

\subsection{Results}

\subsubsection{Stability Classification}

Among \NLIGOOK\ events passing sanity checks, \LIGOFailed\ failed numerical fitting in at least one stability window. Percentages below are computed over all \NLIGOOK\ events passing sanity checks; failed events are excluded from curvature/regression analyses but included in stability tallies:

\begin{center}
\begin{tabular}{lcc}
\toprule
\textbf{Category} & \textbf{N} & \textbf{\% of OK} \\
\midrule
Stable Delayed & \LIGOStableDelayed & \LIGOStableDelayedPct\% \\
Stable Fast & \LIGOStableFast & \LIGOStableFastPct\% \\
Flip & \LIGOFlip & \LIGOFlipPct\% \\
Failed & \LIGOFailed & \LIGOFailedPct\% \\
\bottomrule
\end{tabular}
\end{center}

The flip rate (\LIGOFlipPct\%) indicates a boundary population---events near the classification threshold or with ambiguous fit quality. Per the protocol (Section~\ref{sec:protocol}), flips include both flip-determinate (Fast/Delayed disagreement) and flip-uncertain (at least one Uncertain window) cases; detailed breakdown is reported in follow-up analyses.

\subsubsection{Curvature Separation}

Among \NLIGOStable\ stable events (\LIGOStableDelayed\ Delayed, \LIGOStableFast\ Fast), the curvature index shows strong separation.\footnote{Analyses requiring unique GPS times and valid curvature estimates (e.g., dip tests in Appendix~\ref{app:robustness}) use N=\LIGODipStableN\ after deduplication; the difference does not affect conclusions.}

\begin{center}
\begin{tabular}{lcc}
\toprule
\textbf{Category} & \textbf{Median $b$ (ms$^{-2}$ $\times 10^{3}$)} & \textbf{IQR} \\
\midrule
Stable Delayed & $\LIGOCurvDelayedMedian$ & $[\LIGOCurvDelayedIQRLo, \LIGOCurvDelayedIQRHi]$ \\
Stable Fast & $\LIGOCurvFastMedian$ & $[\LIGOCurvFastIQRLo, \LIGOCurvFastIQRHi]$ \\
\bottomrule
\end{tabular}
\end{center}

Effect sizes:
\begin{itemize}
\item Mann-Whitney $p < 10^{-\LIGOMWCurvPExp}$
\item Cliff's $\delta = \LIGOCliffsDelta$ (large effect) \cite{cliff1993}
\item $\text{AUC}(b) = \LIGOCurvatureAUC$ (bootstrap 95\% CI)
\end{itemize}

Although the stable delayed median is near zero, the distribution is shifted upward relative to fast; discrimination is quantified by AUC$(b)$ and Cliff's $\delta$ rather than by median alone. The curvature index provides strong discrimination between delayed and fast events, with bootstrap confidence intervals excluding chance performance. Note that discrimination is not a hard sign split; it is a distributional shift quantified by AUC and effect size.

\begin{figure}[H]
\centering
\includegraphics[width=0.95\textwidth]{../figures/ligo/curvature_index_plot.png}
\caption{Early-time curvature separation in LIGO glitches. \textbf{Left:} Boxplot of curvature index $b$ by stability category. Stable Delayed events cluster near $b \approx 0$ (flat-then-accelerating recovery), while Stable Fast events show strongly negative $b$ (steep initial recovery). Flip events occupy the boundary region. \textbf{Right:} Density distributions showing clear separation between populations. The curvature index discriminates Delayed from Fast with AUC$(b) = \LIGOCurvatureAUC$. Curvature $b$ is computed on the rising recovery coordinate $z(t) = 1 - (E(t) - E_\infty)/(E_{\text{peak}} - E_\infty)$, where positive $b$ indicates flat-then-accelerating recovery geometry. See Appendix~\ref{app:population} for formal bimodality tests and Appendix~\ref{app:examples} for representative event waveforms.}
\label{fig:ligo_curvature}
\end{figure}

\subsubsection{SNR Control}

A potential concern is that curvature might simply reflect event magnitude (louder events look different). We address this with logistic regression on the \textbf{stable-only subset} (\NLIGOStable\ events: \LIGOStableDelayed\ Stable Delayed, \LIGOStableFast\ Stable Fast; flips excluded). Predictors are standardized (both $b$ and SNR are z-scored within this subset). We report $\Delta\text{BIC} = \text{BIC}(\text{model}) - \text{BIC}(\text{SNR-only})$, so negative values indicate improvement over the SNR-only baseline.

\begin{center}
\begin{tabular}{lcccc}
\toprule
\textbf{Model} & \textbf{$\beta_b$} & \textbf{95\% CI} & \textbf{$\beta_{\text{SNR}}$} & \textbf{$\Delta$BIC} \\
\midrule
Delayed $\sim$ SNR & --- & --- & $\LIGOBetaSNROnly$ & (ref) \\
Delayed $\sim$ $b$ & $\LIGOBetaBOnly$ & --- & --- & $\LIGODeltaBICBOnly$ \\
Delayed $\sim$ $b$ + SNR & $\LIGOBetaB$ & $[\LIGOBetaBLo, \LIGOBetaBHi]$ & $\LIGOBetaSNR$ & $\LIGODeltaBIC$ \\
\bottomrule
\end{tabular}
\end{center}

Bootstrap 95\% CI for $\Delta$BIC (full model vs.\ SNR-only): $[\LIGODeltaBICLo, \LIGODeltaBICHi]$

Key findings:
\begin{enumerate}
\item Curvature alone outperforms SNR alone ($\Delta$BIC $= \LIGODeltaBICBOnly$ for Delayed $\sim b$ vs.\ SNR-only)
\item Curvature coefficient remains strongly positive after SNR control ($\beta_b = +\LIGOBetaB$)
\item SNR coefficient is \textit{negative}: Delayed events are not preferentially loud
\item Bootstrap 95\% CI for $\beta_b$ excludes zero
\item Bootstrap 95\% CI for $\Delta$BIC is entirely negative (curvature robustly outperforms SNR)
\end{enumerate}

Within the analyzed subset, this disfavors a magnitude-only explanation: curvature remains predictive after SNR control, and the SNR coefficient is negative.

\paragraph{Window-level regression (collider-bias check).}
The stable-only regression above conditions on stability---a potential collider. To address this, we ran a window-level logistic regression using \emph{all} determinate window classifications (not just stable events), with standard errors clustered by event ID. This yields \LIGOWindowN\ observations from \LIGOWindowClusters\ unique events. Result: $\beta_b = +\LIGOWindowBetaB$ ($z = \LIGOWindowZB$, $p < 0.00001$), $\beta_{\text{SNR}} = \LIGOWindowBetaSNR$ ($p = \LIGOWindowPSNR$). Higher curvature predicts delayed classification ($\text{OR}_b \approx 5$ per SD) even without conditioning on stability, ruling out collider bias. See Appendix~\ref{app:robustness} for details.

\subsubsection{Flip Direction}

Among flip events, we examined whether Delayed classification appears preferentially at short or long windows. For this tally we include only flips with determinate (non-Uncertain) classifications at both 60~ms and 150~ms (\LIGOFlipDeterminateN/\LIGOFlip):
\begin{itemize}
\item Approximately \LIGOFlipShortDelayedPct\% show Delayed at 60 ms but Fast at 150 ms
\item Approximately \LIGOFlipLongDelayedPct\% show the reverse pattern (Delayed at 150 ms)
\end{itemize}

This asymmetry suggests that hesitation is predominantly a \textit{short-window} phenomenon: among boundary events, the delayed-geometry signature emerges more frequently at 60~ms than at 150~ms. This is consistent with a transient constraint-limited phase that relaxes before the longer window captures the full recovery. The pattern suggests that hesitation manifests early and resolves as the system returns to equilibrium.

\subsection{Interpretation}

\subsubsection{What the Data Show}

Within the analyzable single-pulse subset of LIGO Extremely Loud glitches:
\begin{enumerate}
\item A substantial fraction (\LIGOStableDelayed/\NLIGOOK; \LIGOStableDelayedPct\%) show hesitation-like early-time geometry
\item This is characterized by less-negative / upward-shifted curvature (often near-zero), consistent with flat-then-accelerating onset
\item The signature is robust to window variation (stable core) with a genuine boundary population (flips)
\item Curvature discrimination is not explained by event magnitude
\item \textbf{Null simulation control:} Across window-decoupled null simulations and adversarial parameter sweeps, the analysis pipeline cannot generate a delayed population remotely approaching \LIGOStableDelayedPct\%; the maximum null ceiling is $\leq$\LIGOSweepOverallMaxCIHi\% (worst-case window \LIGOSweepWorstWindow~ms), and spliced-null tests collapse discrimination to chance (Appendix~\ref{app:null-control})
\end{enumerate}

\noindent\textbf{Note on discriminative statistic:} The key metric distinguishing null from reality is the \emph{population fraction} of delayed events (\LIGOStableDelayedPct\% vs.\ $\leq$\LIGOSweepOverallMaxCIHi\% ceiling), not the AUC. High AUC can persist in null simulations because within-trace features remain correlated even when the population is purely fast; what the null cannot produce is a large delayed population.

\subsubsection{Relation to BLQC Predictions}

BLQC predicts that when perturbation dynamics ($\lambda$) exceed observer capacity ($C \ln 2$), recovery exhibits a characteristic ``hesitation'' before normal dynamics resume. The LIGO data are consistent with this prediction:
\begin{itemize}
\item \textbf{Hesitation signature (by protocol):} Delayed-geometry models (sigmoid or delayed exponential) win the AICc tournament under $\Delta\text{AICc} \geq 2$ in all stability windows
\item \textbf{Geometric invariant:} Upward-shifted early-time curvature in Delayed events (near-zero vs.\ strongly negative)
\item \textbf{Threshold behavior:} Two-population structure (stable cores) with a boundary flip set
\end{itemize}
Notably, curvature $b$ is computed separately from the model tournament and still discriminates Stable Delayed vs Stable Fast events. This alignment is consistent with a real geometric difference---not an independent dataset, but a non-definitional check on the same time series.

BLQC does not claim to replace classical control theory understanding of servo dynamics; rather, it proposes that control saturation reflects a \textit{fundamental information-theoretic bound} that applies to any control system, regardless of hardware implementation.

\subsubsection{Caveats}

\begin{enumerate}
\item \textbf{Selection:} Results characterize single-pulse morphologies passing sanity checks, not all glitches
\item \textbf{Sample size:} Current analysis is based on \NLIGOOK\ unique events (see Section~\ref{sec:scope})
\item \textbf{Mechanism:} We observe a pattern consistent with BLQC predictions but do not establish a causal link to specific control loop dynamics
\end{enumerate}

\subsubsection{Seed-Sweep Robustness}

To rule out seed artifacts from the deterministic sampling procedure, we repeated the full LIGO analysis over a sweep of random seeds while holding the sample size and pipeline fixed. Across seeds, the delayed fraction remained stable (33.0\%--35.6\%, SD 0.79\%), discrimination remained high (AUC 0.939--0.958), and unsupervised GMM clustering remained consistent with model-derived labels (agreement 85.6\%--90.3\%). Full results are reported in Appendix~\ref{app:seed-robustness}.

\subsection{Summary of LIGO Findings}

\begin{enumerate}
\item \textbf{Selection funnel:} \LIGOAfterCuts\ population $\rightarrow$ \LIGODedupUnique\ after GPS deduplication $\rightarrow$ \NLIGOOK\ passed sanity + fetch
\item \textbf{Stability (of \NLIGOOK):} \LIGOStableDelayedPct\% stable delayed, \LIGOStableFastPct\% stable fast, \LIGOFlipPct\% flip (boundary population)
\item \textbf{Curvature separation:} AUC$(b) = \LIGOCurvatureAUC$, Cliff's $\delta = \LIGOCliffsDelta$
\item \textbf{SNR control:} $\beta_b = +\LIGOBetaB$ [$\LIGOBetaBLo$, $\LIGOBetaBHi$], $\Delta$BIC $= \LIGODeltaBIC$ [$\LIGODeltaBICLo$, $\LIGODeltaBICHi$]
\item \textbf{Short-window phenomenon:} Among determinate flips, $\sim$\LIGOFlipShortDelayedPct\% show Delayed at short windows (60~ms)
\end{enumerate}

Among analyzable single-pulse LIGO glitches, a substantial subset exhibits hesitation-like recovery dynamics consistent with BLQC predictions, characterized by a robust early-time curvature signature that is not explained by event magnitude. Additional robustness checks---unsupervised GMM validation (Appendix~\ref{app:gmm}), rejected-morphology stress test (Appendix~\ref{app:stress-test}), alternative likelihood (Appendix~\ref{app:alt-likelihood})---are reported in Appendix~\ref{app:robustness}.

% ===== END INLINED: ligo_virgo.tex =====

% ===== BEGIN INLINED: cross_system.tex =====
% ============================================================
% SECTION 6: CROSS-SYSTEM COMPARISON
% ============================================================
\section{Cross-System Comparison}
\label{sec:cross}

\subsection{Two Modes of Hesitation}

BLQC predicts that when perturbation dynamics ($\lambda$) exceed observer capacity ($C \ln 2$), recovery exhibits characteristic hesitation. In our operationalization, this hesitation manifests in \textit{two distinct modes}:

\begin{enumerate}
\item \textbf{Delay mode ($t_{\mathrm{inf}} > 0$):} The time of maximal recovery slope occurs after the event extremum; empirically this appears as a plateau or lag before rapid recovery. We summarize this by $D = t_{\mathrm{inf}}/\tau$, where $t_{\mathrm{inf}}$ is obtained from the winning model (Section~\ref{sec:protocol}) or from a derivative diagnostic (reported only as a check).
\item \textbf{Curvature mode ($b$ elevated):} Onset may be near-immediate ($D \approx 0$), but the first $\sim$20~ms exhibits flat-then-accelerating recovery. Characterized by elevated (shifted upward) early-time curvature index $b$ relative to fast events.
\end{enumerate}

These modes are observational summaries, not classification rules. They are not mutually exclusive but represent different projections of the same underlying phenomenon. The mode that dominates depends on platform-specific factors: sampling rate, noise floor, and controller architecture.

\subsection{Platform Comparison}

Table~\ref{tab:comparison} summarizes hesitation detection across platforms. \textbf{Methodological note:} Quantitative cross-platform comparison (stable-core percentages, curvature metrics) applies to 26-qubit and LIGO, which share the same 3-window stability protocol. 63-qubit data serve as a qualitative baseline/contrast illustration only, due to different observables and pipeline constraints (see footnote).

\begin{table}[H]
\centering
\caption{Cross-platform hesitation signatures. Classification is defined by model geometry (delayed vs fast); $b$ aligns with the classification where discrimination is present.}
\label{tab:comparison}
\footnotesize
\begin{tabular}{lcccll}
\toprule
\textbf{Platform} & \textbf{N} & \textbf{Stable Del.} & \textbf{Stable Fast} & \textbf{Defining} & \textbf{Classifier-external} \\
\midrule
63-qubit (Li)$^\dagger$ & 92 & --- & Fast (qual.) & Qualitative & Baseline (capacity-wins) \\
26-qubit (McEwen) & \NMcEwen & \McEwenStableDelayedPct\% & \McEwenStableFastPct\% & Model geom. & $b$: NS ($p = \McEwenCurvatureP$) \\
LIGO H1 O3a & \NLIGOOK & \LIGOStableDelayedPct\% & \LIGOStableFastPct\% & Model geom. & $b$: AUC $= \LIGOCurvatureAUC$ \\
\bottomrule
\end{tabular}

\smallskip
\scriptsize $^\dagger$\emph{Not subjected to 3-window stability protocol; treated as qualitative baseline due to pipeline differences (10 CPJ + 82 bit-flip events, all fast recovery).}
\end{table}

\subsection{The Hesitation Phase Diagram}

The hesitation phase space is spanned by two dimensions: delay ($D$) and curvature ($b$). (Curvature is computed on the normalized recovery variable $z(t)$, which increases from 0 at the event extremum to 1 at baseline.) Events can be classified by their position in this space:

\begin{itemize}
\item \textbf{Fast recovery:} $D \approx 0$, $b < 0$ (steep initial recovery that decelerates)
\item \textbf{Delay-dominant hesitation:} $D$ large, $b \approx 0$ (clear plateau then recovery)
\item \textbf{Curvature-dominant hesitation:} $D$ small but $> 0$, $b$ shifted upward (flat-then-accelerating geometry provides discrimination)
\item \textbf{Mixed hesitation:} $D$ large, $b$ elevated (both signatures present)
\end{itemize}

Note: $D$ is defining (separation is by construction; see Section~\ref{sec:protocol}); $b$ is corroborating. The distinction between ``delay-dominant'' and ``curvature-dominant'' refers to which coordinate provides stronger discrimination, not to whether $D$ is literally zero.

BLQC predicts that all hesitation events---regardless of mode---share the same underlying cause: the observer's capacity being temporarily exceeded by the perturbation's information rate. Accordingly, cross-platform agreement on $D$ is methodological; the nontrivial cross-platform evidence is the presence or absence of additional geometric separation (e.g., $b$) and its robustness to confounds.

\begin{figure}[H]
\centering
\includegraphics[width=0.95\textwidth]{../figures/hesitation_phase_diagram.png}
\caption{Hesitation phase diagram showing two modes across platforms. Horizontal axis: delay fraction $D = t_{\mathrm{inf}}/\tau$. Vertical axis: early-time curvature index $b$ (ms$^{-2}$ $\times 10^{3}$, measured on first 20~ms). \textbf{McEwen Delayed} (red circles) shows high delay but normal curvature---hesitation expressed primarily along $D$. \textbf{LIGO Delayed} (red squares) shows elevated curvature---hesitation expressed through early-time geometry. Standard events from both platforms cluster at $D \approx 0$ with negative curvature. The dashed line at $b = 0$ separates concave-up vs.\ concave-down early-time geometry; in McEwen, $b$ does not discriminate (both populations overlap), while in LIGO curvature provides additional separation beyond the defining geometry-based classification.}
\label{fig:phase_diagram}
\end{figure}

\subsection{Platform-Dependent Expression}

Why do different platforms express hesitation in different modes?

\paragraph{26-Qubit Processor (delay-dominant):}
The calibration telemetry has 100~$\mu$s sampling (10 kHz) with discrete integer counts and moderate noise. The model-implied onset delay ($D$) is directly resolvable. The small stable delayed population (\McEwenStableDelayed/\NMcEwen; \McEwenStableDelayedPct\%) shows delayed onset before recovery, making delay the natural classification feature. Notably, the early-time curvature index $b$ does \textit{not} discriminate between delayed and fast events in this platform ($p = \McEwenCurvatureP$, NS)---once recovery begins, both populations show similar curvature. Hesitation is encoded entirely in the delay parameter.

\paragraph{LIGO interferometer (curvature-dominant):}
Strain data (4 kHz, sufficient for 0--150~ms envelope features) has high broadband noise. In LIGO, hesitation is expressed primarily through early-time geometry: the curvature index $b$ provides strong discrimination (AUC $= \LIGOCurvatureAUC$) between stable delayed and stable fast populations, and this separation survives SNR control.

\paragraph{63-Qubit Processor (baseline):}
In the high-resolution channel (10 CPJ bursts) and slower bit-flip channel (82 events), we observe uniformly fast recovery consistent with a capacity-wins baseline in this platform. Due to observable and pipeline differences, we treat this as a qualitative baseline rather than applying the same stability analysis.

\subsection{Effect Size Comparison}

$D$ is defining; $b$ is corroborating (see Section~\ref{sec:protocol}):

\begin{center}
\footnotesize
\begin{tabular}{llll}
\toprule
\textbf{Platform} & \textbf{Defining} & \textbf{Corroborator} & \textbf{Summary} \\
\midrule
26-qubit & Model geometry & $b$ & NS (MW $p = \McEwenCurvatureP$) \\
LIGO & Model geometry & $b$ & AUC $= \LIGOCurvatureAUC$; $\beta_b > 0$ (SNR-controlled) \\
\bottomrule
\end{tabular}
\end{center}

The scientifically meaningful result: in LIGO, curvature $b$ provides strong classifier-external discrimination that survives SNR control ($\beta_b$ CI excludes zero); in 26-qubit data it does not.

\subsection{The Threshold Interpretation}

A key feature of BLQC is the \textit{sharp threshold} at $C^* = \lambda / \ln 2$. This framework is consistent with:

\begin{itemize}
\item \textbf{Stable-core populations:} Events cluster into distinct stable delayed and stable fast populations separated by recovery geometry (see Appendix~\ref{app:population} for unimodality tests supporting multi-cluster structure)
\item \textbf{Boundary region:} A minority of events (flip/uncertain) classify inconsistently across analysis variations, suggesting they lie near the decision threshold
\item \textbf{Possible severity dependence:} Evidence for correlation between perturbation magnitude and hesitation is suggestive in some datasets; formal quantification is deferred to prospective protocols
\end{itemize}

Linear systems would show continuous variation in recovery dynamics. The observed population structure---departure from unimodality when pooled, unimodal when separated---is consistent with two-regime dynamics driven by some nonlinearity. This interpretation is consistent with BLQC but not uniquely diagnostic; alternative nonlinearities in physical relaxation or control loops could also produce two-population structure.

\subsection{Implications}

\paragraph{1. Cross-platform presence:} Hesitation-like signatures appear across very different physical systems (superconducting qubits, optomechanical interferometers). The \textit{existence} of hesitation-like signatures appears cross-platform; its \textit{expression} is platform-dependent.

\paragraph{2. Feature selection guidance:} When searching for BLQC signatures in new systems, both delay ($D$) and curvature ($b$) should be examined. The dominant mode depends on the system's noise characteristics and sampling resolution.

\paragraph{3. Falsifiability:} A system with known high capacity ($C \gg \lambda / \ln 2$) should show 0\% delayed-geometry fraction regardless of perturbation magnitude. The 63-qubit data (uniformly fast across 10 CPJ + 82 bit-flip events) are consistent with this prediction.

\paragraph{4. Design implications:} If hesitation reflects an information-theoretic bound, mitigation may require either increasing effective capacity $C$ or reducing effective instability $\lambda$; purely local engineering changes may shift these parameters but may not remove the phenomenon entirely.

% ===== END INLINED: cross_system.tex =====

% ===== BEGIN INLINED: discussion.tex =====
% ============================================================
% SECTION 7: DISCUSSION
% ============================================================
\section{Discussion}
\label{sec:discussion}

\subsection{Alternative Explanations}

We consider potential alternative explanations for the observed hesitation signatures (delayed-geometry onset and/or early-time curvature anomalies):

\paragraph{Thermal relaxation effects:} Standard single-timescale quasiparticle relaxation (pure exponential or power-law) can explain variation in recovery \emph{timescale}, but does not by itself explain a \emph{stable population structure} in which a minority repeatedly exhibits delayed-geometry onset or early-time curvature anomalies while the majority remains immediate-onset under the same preprocessing and fitting protocol. Explaining this structure requires an additional threshold-like mechanism---whether BLQC or some alternative control/physical nonlinearity.

\paragraph{Measurement artifacts:} Our primary classifications come from a parametric model tournament with stability assessed across multiple window lengths. Derivative-based peak timing is retained only as a diagnostic. The stable-core populations persist across reasonable analysis variations, arguing against a pure preprocessing artifact. If hesitation were a uniform artifact, it should appear equally in all events, not in a stable minority. Crucially, in LIGO the curvature statistic $b$ is computed separately from the AICc geometry label and still separates stable delayed vs.\ stable fast under SNR control, providing non-circular consistency with the classification.

\paragraph{Detector saturation:} Photodetector or ADC saturation could cause apparent plateaus. However, this should correlate with signal \textit{amplitude}, not with the time-domain recovery dynamics we classify. Moreover, saturation often produces clipping, not smooth transitions with delayed-geometry onset.

\paragraph{Filter/envelope effects:} For LIGO we use the Hilbert envelope specifically to avoid $|$strain$|$ zero-crossing cusps. While envelope extraction and filtering can affect morphology, the key result is that curvature separation persists across window sweeps and is not explained by SNR, which argues against a trivial magnitude- or processing-induced artifact.

\paragraph{Preprocessing robustness (LIGO):} We explicitly tested sensitivity to: (i)~bandpass choices (10--500~Hz nominal; nearby variants did not alter headline results), (ii)~envelope method (Hilbert vs.\ absolute value), (iii)~baseline definition $E_\infty$ (terminal median), and (iv)~peak-alignment definition (envelope maximum). The curvature separation and stable-core structure persisted across these variations. We report this to address the skeptical line that ``your curvature split is a filter/envelope artifact.''

\paragraph{Statistical fluctuations:} Random noise could occasionally produce delayed model fits. However, the stability analysis (classifying across multiple windows) filters out noise-driven classifications. The stable delayed populations in both platforms represent consistent signatures---supported by Hartigan's dip test indicating departure from unimodality in the pooled distribution while each subpopulation is individually consistent with unimodality (Appendix~\ref{app:diptest}).

We are not aware of a single processing artifact that naturally reproduces the joint pattern: stable-core populations in both platforms, boundary/flip events near the classification threshold, cross-platform agreement that ``hesitation exists'' with different discriminating features, and (in LIGO) curvature separation surviving SNR control.

\paragraph{The asymmetry argument:} A key point is that the BLQC label is not defined by curvature: it is defined by a delayed-geometry model winning an AICc tournament. In 26-qubit data, this produces a delay-dominant hesitation population with no curvature separation; in LIGO, a non-definitional curvature statistic (computed separately from the model tournament) separates stable delayed vs stable fast even under SNR control. This asymmetry is difficult to explain as a single preprocessing artifact.

\subsection{Epistemic Status}

We emphasize the limitations of this analysis:

\paragraph{Retrospective, not prospective:} These analyses were not preregistered; the methodology was developed with knowledge of the data.

\paragraph{Exploratory, not confirmatory:} The results establish \textit{plausibility}, not proof. They show that BLQC-consistent signatures exist in real data, motivating controlled experiments.

\paragraph{Correlation, not causation:} We observe recovery dynamics consistent with BLQC predictions, but we cannot exclude unknown confounding factors without controlled manipulation.

\paragraph{Sample sizes:} The 26-qubit analysis (\NMcEwen\ events) and LIGO analysis (\NLIGOOK\ unique events after GPS deduplication) provide reasonable statistics for stable-core classification. The LIGO results are based on a specific morphology class (single-pulse events passing sanity checks); generalization to other glitch types requires further investigation.

\paragraph{Selection funnel:} For LIGO, we report the full selection funnel (metadata $\to$ fetched strain $\to$ matched $\to$ passed morphology), and interpret results as conditional on single-pulse morphologies. This transparency allows readers to assess generalizability.

The appropriate conclusion is: \textit{these findings justify investment in prospective controlled experiments}, not \textit{the BLQC mechanism is proven}.

\subsection{Prospective Test}

The key test that distinguishes BLQC from hardware-specific explanations is manipulation of effective control bandwidth. If BLQC is correct, increasing effective bandwidth (faster feedback, higher sampling rate, reduced latency) should move boundary events from delayed to fast geometry; decreasing bandwidth should do the reverse. If the population structure instead reflects hardware nonlinearity, it should persist regardless of bandwidth manipulation.

This prediction---that varying bandwidth causes systematic migration across the fast/delayed boundary---is what distinguishes the BLQC interpretation from purely hardware-specific explanations and motivates prospective controlled experiments.

% ===== END INLINED: discussion.tex =====

% ===== BEGIN INLINED: conclusion.tex =====
% ============================================================
% SECTION 8: CONCLUSION
% ============================================================
\section{Conclusion}
\label{sec:conclusion}

We have documented within-platform heterogeneity in recovery dynamics across three independent experimental systems: superconducting qubits (26-qubit Sycamore and 63-qubit Chinese processors) and LIGO gravitational-wave interferometers. The central finding is population structure---events cluster into distinct geometries (fast vs.\ delayed onset) with a boundary subset---rather than continuous variation. The same two-regime pattern appears across platforms as different as superconducting qubits and km-scale interferometers.

The consistency of this pattern---stable-core populations, boundary events near classification thresholds, curvature signatures surviving SNR control---is difficult to explain as a single preprocessing artifact. The BLQC framework predicts exactly this structure: a threshold separating capacity-wins (immediate recovery) from chaos-wins (delayed-onset hesitation) regimes.

This work is retrospective and exploratory: it establishes plausibility, not proof. The appropriate conclusion is that BLQC-consistent signatures exist in current precision-instrument data and warrant prospective controlled experiments that manipulate effective control bandwidth to test the causal mechanism.

% ===== END INLINED: conclusion.tex =====


% ------------------------------------------------------------
% BIBLIOGRAPHY
% ------------------------------------------------------------
\newpage
\section*{References}
\addcontentsline{toc}{section}{References}

\renewcommand{\refname}{}
\vspace{-2em}
\begin{thebibliography}{99}

\bibitem{dekker2025blqc}
Dekker, A. (2025).
\newblock A Control-Theoretic Approach to Quantum Measurement: Bandwidth Limits on Quantum Control.
\newblock OSF Preprints. \href{https://doi.org/10.17605/OSF.IO/G5WRH}{doi:10.17605/OSF.IO/G5WRH}

\bibitem{mcewen2022}
McEwen, M., et al. (2022).
\newblock Resolving catastrophic error bursts from cosmic rays in large arrays of superconducting qubits.
\newblock \textit{Nature Physics}, 18, 107--111.
\newblock \href{https://doi.org/10.1038/s41567-021-01432-8}{doi:10.1038/s41567-021-01432-8}

\bibitem{li2025}
Li, X., et al. (2025).
\newblock Cosmic-ray-induced correlated errors in superconducting qubit array.
\newblock \textit{Nature Communications}, 16, 4677. arXiv:2402.04245.
\newblock \href{https://doi.org/10.1038/s41467-025-59778-z}{doi:10.1038/s41467-025-59778-z}

\bibitem{gravityspy}
Zevin, M., et al. (2017).
\newblock Gravity Spy: integrating advanced LIGO detector characterization, machine learning, and citizen science.
\newblock \textit{Classical and Quantum Gravity}, 34(6), 064003.

\bibitem{gwosc}
LIGO Scientific Collaboration, Virgo Collaboration, and KAGRA Collaboration (2023).
\newblock GWOSC: Gravitational Wave Open Science Center.
\newblock \url{https://gwosc.org/}

\bibitem{burnham2002}
Burnham, K.P., \& Anderson, D.R. (2002).
\newblock \textit{Model Selection and Multimodel Inference: A Practical Information-Theoretic Approach} (2nd ed.).
\newblock Springer.

\bibitem{hartigan1985}
Hartigan, J.A., \& Hartigan, P.M. (1985).
\newblock The dip test of unimodality.
\newblock \textit{The Annals of Statistics}, 13(1), 70--84.

\bibitem{cliff1993}
Cliff, N. (1993).
\newblock Dominance statistics: Ordinal analyses to answer ordinal questions.
\newblock \textit{Psychological Bulletin}, 114(3), 494--509.

\end{thebibliography}


% ------------------------------------------------------------
% APPENDICES
% ------------------------------------------------------------
\newpage
\appendix
% ===== BEGIN INLINED: appendices.tex =====
% ============================================================
% APPENDICES
% ============================================================

\section{Analysis Scripts and Reproducibility}
\label{app:scripts}

\noindent\textit{Provenance: \ResultsBanner}
\bigskip

All analysis scripts are included in the \texttt{scripts/} directory of this OSF project (\href{https://doi.org/10.17605/OSF.IO/M5TB9}{doi:10.17605/OSF.IO/M5TB9}). This appendix provides step-by-step instructions for reproducing all results from a clean environment.

\subsection{System Requirements}

\begin{itemize}
\item \textbf{Python:} 3.9 or later (tested with 3.11)
\item \textbf{Disk space:} Large download for full LIGO catalog (order of 100+ GB); modest for cached mode
\item \textbf{Network:} Required for initial data fetch; cached mode works offline
\item \textbf{Time:} Full pipeline may take significant time (dominated by data retrieval); cached mode is substantially faster
\item \textbf{LaTeX:} pdflatex (TeX Live or equivalent) for PDF compilation
\item \textbf{Git:} For cloning the repository
\end{itemize}

\subsection{Quick Start (Complete Reproduction)}

The entire analysis can be reproduced with four commands:

\begin{verbatim}
# 1. Clone the repository
git clone https://github.com/aernouddekker/forensic-signatures.git
cd forensic-signatures

# 2. Create virtual environment and install dependencies
python3 -m venv venv
source venv/bin/activate  # Linux/macOS
pip install -r requirements.txt

# 3. Run the complete pipeline (downloads all data automatically)
cd scripts
python run_all.py --clean --n_events 500  # ~30 GB download, ~30 min

# 4. Output: latex/signatures_combined.pdf
\end{verbatim}

The \texttt{--clean} flag creates all required directories and downloads all datasets fresh. The \texttt{run\_all.py} script automatically downloads from GWOSC (LIGO) and Figshare (McEwen, Li et al.), runs all analysis steps, generates LaTeX macros, and compiles the PDF.

\subsection{Environment Setup (Detailed)}

For manual setup or troubleshooting:

\begin{verbatim}
# Create and activate virtual environment
python3 -m venv forensic_env
source forensic_env/bin/activate  # Linux/macOS
# or: forensic_env\Scripts\activate  # Windows

# Install dependencies
pip install -r requirements.txt
\end{verbatim}

\noindent The \texttt{requirements.txt} specifies pinned versions for reproducibility:

\begin{verbatim}
# Core scientific computing
numpy==1.26.4
scipy==1.13.1
pandas==2.3.3
matplotlib==3.9.4

# Statistical tests
diptest==0.10.0  # Hartigan's dip test

# LIGO data access
gwpy==3.0.13
gwosc>=0.7.1

# Optional
tqdm>=4.60  # Progress bars
\end{verbatim}

\subsection{Data Sources}

The pipeline automatically downloads all required datasets. No manual data download is needed.

\paragraph{LIGO Glitch Metadata (auto-downloaded):}
\begin{itemize}
\item Source: Gravity Spy O3a catalog
\item Zenodo DOI: \href{https://doi.org/10.5281/zenodo.5649212}{10.5281/zenodo.5649212}
\item Downloaded automatically by \texttt{ligo\_bulk\_download.py}
\end{itemize}

\paragraph{LIGO Strain Data (auto-downloaded):}
\begin{itemize}
\item Source: GWOSC (\url{https://gwosc.org/})
\item Downloaded via \texttt{gwpy}/\texttt{gwosc} libraries
\item Smart bulk download: fetches unique 4096-second HDF5 files containing events
\end{itemize}

\paragraph{26-Qubit Processor (auto-downloaded):}
\begin{itemize}
\item Source: McEwen et al. (2022) supplementary data
\item Figshare DOI: \href{https://doi.org/10.6084/m9.figshare.16673851}{10.6084/m9.figshare.16673851}
\item Downloaded automatically to \texttt{scripts/data/mcewan/}
\end{itemize}

\paragraph{63-Qubit Processor (auto-downloaded):}
\begin{itemize}
\item Source: Li et al. (2025) supplementary data
\item Figshare DOI: \href{https://doi.org/10.6084/m9.figshare.28815434}{10.6084/m9.figshare.28815434}
\item Downloaded automatically to \texttt{scripts/data/chinese/}
\end{itemize}

\subsection{One-Command Reproduction}

The master script \texttt{run\_all.py} executes the complete pipeline:

\begin{verbatim}
cd scripts

# Fresh start (first run) - creates directories, downloads all data
python run_all.py --clean --n_events 500  # ~30 GB, ~30 min

# Quick test with minimal data
python run_all.py --clean --n_events 10   # ~1 GB, ~5 min

# Use cached data (requires prior run)
python run_all.py --cached

# Regenerate macros and PDF only (assumes data exists)
python run_all.py --macros
\end{verbatim}

\noindent The script automatically:
\begin{enumerate}
\item Downloads LIGO bulk data (if not cached)
\item Runs all 15 analysis steps in dependency order
\item Generates \texttt{results\_macros.tex} from JSON outputs
\item Combines LaTeX source files
\item Compiles the PDF (if pdflatex is available)
\end{enumerate}

\subsection{Pipeline Steps (Manual Execution)}

For debugging or selective re-runs, individual scripts can be executed:

\begin{verbatim}
# === LIGO Analysis ===
# 1. Bulk data download (~168 GB for full catalog)
python ligo_bulk_download.py --class Extremely_Loud \
    --output bulk_data

# 2. Primary analysis (envelope extraction, model fitting)
python ligo_glitch_analysis.py --classes Extremely_Loud \
    --bulk_data_dir bulk_data

# 3. 3-window stability classification
python ligo_stability_figures.py

# 4-6. Robustness checks
python ligo_curvature_sweep.py    # Fit interval sensitivity
python ligo_threshold_sweep.py   # AICc threshold sensitivity
python ligo_baseline_robustness.py

# 7. Appendix figures (dip test, event examples)
python ligo_appendix_figures.py

# 8-10. Validation analyses
python ligo_gmm_validation.py      # Unsupervised validation
python ligo_negative_control.py    # Rejected-morphology test
python ligo_alt_likelihood.py      # Log-domain robustness

# 11. Null simulation control
python ligo_null_simulation.py --mode both --n_synthetic 1000

# === 26-Qubit Analysis ===
# 12. Event extraction
python google_mcewan_analysis.py --freeze-only

# 13-15. Stability analysis
python google_robustness_sweep.py
python google_stability_diagnostics.py
python mcewan_tau_stats.py

# === Figure Generation ===
python hesitation_phase_diagram.py

# === Macro Generation ===
python generate_macros.py --strict

# === LaTeX Compilation ===
cd ../latex
python combine_latex.py
pdflatex signatures_combined.tex
pdflatex signatures_combined.tex  # Second pass for references
\end{verbatim}

\subsection{Output Structure}

After a successful run, the directory structure is:

\begin{verbatim}
scripts/
  output/
    ligo_envelope/
      bootstrap_beta_b.json         # Main LIGO results
      stability_events.jsonl        # Per-event classifications
      curvature_sweep_results.json  # Window robustness
      threshold_sweep_results.json  # AICc sensitivity
      dip_test.json                 # Unimodality tests
      unsupervised_validation.json  # GMM validation
      null_simulation.json          # Null control results
    robustness_sweep/
      stability_table.json          # McEwen stability counts
      stability_diagnostics.json    # Evidence strength
      mcewan_tau_stats.json         # Recovery timescales
  bulk_data/                        # LIGO HDF5 files (large)
    file_index.txt                  # Index of downloaded files

figures/
  ligo/                             # LIGO figures
  google/                           # McEwen figures
  appendix/                         # Appendix figures

latex/
  results_macros.tex                # Generated macros
  signatures_combined.tex           # Combined document
  signatures_combined.pdf           # Final PDF
\end{verbatim}

\subsection{Verification}

To verify successful reproduction:

\begin{enumerate}
\item Check that \texttt{generate\_macros.py --strict} completes without errors (validates all required JSON files exist)
\item Confirm the PDF compiles without ``undefined macro'' warnings
\item Key metrics to verify against published values:
\begin{itemize}
\item LIGO stable delayed fraction: \LIGOStableDelayedPct\%
\item LIGO curvature AUC: \LIGOCurvatureAUC
\item McEwen stable delayed count: \McEwenStableDelayed
\end{itemize}
\end{enumerate}

\subsection{Reproducibility Notes}

\begin{itemize}
\item \textbf{RNG seed:} All bootstrap confidence intervals use \texttt{np.random.seed(42)} for determinism
\item \textbf{Pipeline version:} Results correspond to pipeline version 1.0
\item \textbf{LIGO data:} Full catalog analysis requires a large download; the \texttt{--n\_events} parameter can limit analysis to top N events by SNR for faster testing
\item \textbf{Network dependency:} Initial LIGO data fetch requires internet; subsequent runs can use \texttt{--cached}
\item \textbf{Platform independence:} Tested on macOS and Linux; Windows may require path adjustments
\end{itemize}

\paragraph{Troubleshooting:}
\begin{itemize}
\item \textbf{gwpy import error:} Ensure Python $\geq$ 3.9; try \texttt{pip install --upgrade gwpy}
\item \textbf{GWOSC timeout:} Network issues; retry or use \texttt{--cached} mode
\item \textbf{Missing JSON files:} Run the prerequisite analysis script(s) first
\item \textbf{Macros showing ``???'':} Source JSON file missing; check \texttt{scripts/output/}
\end{itemize}

\newpage
\section{Population Structure Diagnostics}
\label{app:population}

\subsection{Hartigan's Dip Test}
\label{app:diptest}

To test whether the stable-core curvature distributions support distinct populations (not a single distribution with outliers), we apply Hartigan's dip test for unimodality \cite{hartigan1985}. The dip test assesses departure from unimodality but does not uniquely prove bimodality; results should be interpreted as supportive evidence, not definitive proof.

\begin{figure}[H]
\centering
\includegraphics[width=\textwidth]{../figures/appendix/dip_test_curvature.png}
\caption{Hartigan's dip test on curvature index $b$ for LIGO stable events (N=\LIGODipStableN\ after deduplication). \textbf{Left:} Combined distribution of all stable events (delayed + fast) shows significant departure from unimodality (dip $= \LIGODip$, p = \LIGODipStableP), supporting at least two clusters. \textbf{Center/Right:} Each population individually is consistent with unimodality ($p_{\text{delayed}} = \LIGODipPDelayed$, $p_{\text{fast}} = \LIGODipPFast$), consistent with separation of the two labeled stable populations.}
\label{fig:dip_test}
\end{figure}

The dip test results are diagnostic of clustering structure:
\begin{itemize}
\item The \emph{combined} stable-event distribution rejects unimodality (p = \LIGODipStableP), consistent with at least two clusters.
\item Each population \emph{individually} is consistent with unimodality ($p_{\text{delayed}} = \LIGODipPDelayed$, $p_{\text{fast}} = \LIGODipPFast$), suggesting well-separated clusters rather than a single distribution with a tail.
\end{itemize}

This pattern---departure from unimodality when pooled, unimodal when separated---is consistent with two-population structure. Note: the dip test is descriptive; it cannot adjudicate whether the split reflects a causal mechanism or is partly induced by the classification procedure itself. The test confirms that labeling captures a genuine distributional feature, not that the feature is causally independent of the classifier. Non-significance within subsets should be read as ``consistent with unimodality at available sample size,'' not as evidence against residual substructure.

\newpage
\section{Representative Event Examples}
\label{app:examples}

Examples were selected post hoc to illustrate typical morphology; selection does not affect reported statistics.

\begin{figure}[H]
\centering
\includegraphics[width=\textwidth]{../figures/appendix/event_examples.png}
\caption{Representative LIGO glitch examples by stability category. Each panel shows the Hilbert envelope normalized to $[0,1]$ (peak to baseline), with 3-window classification results annotated. \textbf{Top row:} Stable Fast events show immediate decay from peak (negative curvature). \textbf{Middle row:} Stable Delayed events show hesitation-like geometry (near-zero or positive early curvature). \textbf{Bottom row:} Flip events show intermediate behavior, classified differently across window sizes. Note: Curvature index $b$ is computed on the rising recovery coordinate $z(t) = 1 - (E(t) - E_\infty)/(E_{\text{peak}} - E_\infty)$. Positive $b$ indicates early-time concave-up recovery in $z(t)$ (accelerating recovery: flat-then-steep), corresponding to an envelope decay that starts relatively flat and then steepens.}
\label{fig:event_examples}
\end{figure}

\newpage
\section{Event Catalogs}
\label{app:catalogs}

\paragraph{26-Qubit Events (\NMcEwen\ detected):}
Complete event-by-event classification is available in the analysis output, including:
\begin{itemize}
\item Event timestamp and source file
\item Crash depth (normalized minimum)
\item Winning model and analytic $t_{\mathrm{inf}}$, $\tau$, $D = t_{\mathrm{inf}}/\tau$
\item Stability classification (Stable Delayed / Stable Fast / Flip)
\end{itemize}

Classification summary: \McEwenStableFast\ stable fast (\McEwenStableFastPct\%), \McEwenStableDelayed\ stable delayed (\McEwenStableDelayedPct\%), \McEwenFlip\ flip (\McEwenFlipPct\%).

\paragraph{LIGO Events (\NLIGOOK\ analyzable):}

From \LIGOAfterCuts\ H1 O3a ``Extremely Loud'' glitches (SNR $\geq 50$, confidence $\geq 0.8$), \LIGODedupUnique\ unique events remain after GPS deduplication. Of these, \NLIGOOK\ passed sanity checks and were analyzed; \LIGOFailed\ failed numerical fitting in at least one stability window (see Section~\ref{sec:ligo} for methodology). Classification by 3-window stability (60/100/150 ms):

\begin{center}
\begin{tabular}{lcc}
\toprule
\textbf{Category} & \textbf{N} & \textbf{\%} \\
\midrule
Stable Delayed & \LIGOStableDelayed & \LIGOStableDelayedPct\% \\
Stable Fast & \LIGOStableFast & \LIGOStableFastPct\% \\
Flip (boundary) & \LIGOFlip & \LIGOFlipPct\% \\
Failed & \LIGOFailed & \LIGOFailedPct\% \\
\bottomrule
\end{tabular}
\end{center}

\textbf{Curvature separation:} Among \NLIGOStable\ stable events, the early-time curvature index $b$ provides strong discrimination (AUC $= \LIGOCurvatureAUC$ [\LIGOCurvatureAUClo, \LIGOCurvatureAUChi]) between delayed and fast populations.

Full event-by-event output available in \texttt{output/ligo/}.

\newpage
\section{Robustness and Sensitivity Checks}
\label{app:robustness}

To ensure results are not artifacts of specific methodological choices, we performed the following sensitivity checks.

\paragraph{Summary (LIGO).} We performed six referee-facing checks designed to break label dependence, collider-bias, and pipeline-artifact explanations. (i)~\textit{Unlabeled unimodality:} curvature bimodality is significant in the stable-core subset (dip p = \LIGODipStableP; p = \LIGODipStableTrimP\ after trimming) but not in the pooled stable+flip cohort (dip $p = \LIGODipUnlabeledP$), consistent with flips forming a boundary band that fills the inter-mode gap. (ii)~\textit{No stability-conditioning:} using window-level outcomes (\LIGOWindowN\ observations across \LIGOWindowClusters\ events) with cluster-robust SEs, curvature strongly predicts delayed classification ($\beta_b = +\LIGOWindowBetaB$, $z = \LIGOWindowZB$, $p < 10^{-6}$) while SNR is negative ($\beta_{\text{SNR}} = \LIGOWindowBetaSNR$, $p = \LIGOWindowPSNR$). (iii)~\textit{Unsupervised recovery:} a 2-component GMM fit to curvature alone (no labels) recovers the stable-core split with \LIGOGMMAgreement\% agreement (ARI~$=\LIGOGMMARI$; NMI~$=\LIGOGMMNMI$), supporting a label-independent geometric population structure. (iv)~\textit{Rejected-morphology stress test:} running the same tournament on complex events (rejected by selection criteria) yields AUC$(b) = \LIGORejectAUC$ ($N_{\text{det}} = \LIGORejectDeterminateN$), suggesting the association is not uniquely explained by the morphology filter; the filter's primary role is to stabilize estimation. (v)~\textit{Alternative likelihood:} re-running classification on $\log E(t)$ yields \LIGOAltLikelihoodAgreement\% agreement with original labels, showing results are robust to Gaussian-SSE misspecification. (vi)~\textit{Null simulation control:} in a window-decoupled synthetic test (labels from 30--100~ms, curvature from 0--20~ms, baseline from 110--150~ms), the false-delayed ceiling is \LIGONullMislabelRate\% [CI: \LIGONullMislabelCILo--\LIGONullMislabelCIHi\%], compared to \LIGOStableDelayedPct\% in real LIGO---a gap that cannot be explained by pipeline noise.

\subsection{Derivative Timing as Diagnostic}
\label{app:derivative}

For completeness, we also report derivative-based timing:
\begin{equation}
t_{\text{deriv}} = \arg\max_t \left| \frac{dy}{dt} \right|
\end{equation}
computed on Savitzky--Golay smoothed data (smoothing parameters specified per platform). However, this metric is \textbf{not used for classification} because it is sensitive to smoothing bandwidth. We report it only as a diagnostic to confirm that model-based and derivative-based approaches give consistent orderings when the latter is stable.

\subsection{AICc Threshold Sensitivity}
\label{app:threshold}

The main analysis uses $|\Delta\text{AICc}| \geq 2$ to distinguish Delayed from Fast. We reran classification with thresholds 1 (weaker evidence) and 4 (stronger evidence):

\begin{center}
\begin{tabular}{lcccc}
\toprule
\textbf{Threshold} & \textbf{Stable Delayed} & \textbf{Stable Fast} & \textbf{Flip} & \textbf{AUC($b$)} \\
\midrule
$|\Delta\text{AICc}| \geq 1$ & \LIGOThreshOneDelayedPct\% & \LIGOThreshOneFastPct\% & \LIGOThreshOneFlipPct\% & \LIGOThreshOneAUC \\
$|\Delta\text{AICc}| \geq 2$ (default) & \LIGOStableDelayedPct\% & \LIGOStableFastPct\% & \LIGOFlipPct\% & \LIGOCurvatureAUC \\
$|\Delta\text{AICc}| \geq 4$ & \LIGOThreshFourDelayedPct\% & \LIGOThreshFourFastPct\% & \LIGOThreshFourFlipPct\% & \LIGOThreshFourAUC \\
\bottomrule
\end{tabular}
\end{center}

The stable-delayed fraction remains $\approx$33--34\% across all thresholds. Higher thresholds shift events from stable-fast to flip (more windows become ``uncertain''), but the curvature separation AUC remains high (\LIGOCurvatureAUC--\LIGOThreshFourAUC), indicating the curvature-based separation is robust to threshold choice. The classification boundary, not the curvature signature, is threshold-sensitive.

\subsection{Flip Event Breakdown}
\label{app:flip}

Flip events (3-window disagreement) divide into two categories:
\begin{itemize}
\item \textbf{Determinate:} All three windows yield confident classifications, but they disagree (e.g., 2 delayed + 1 fast). These are genuine boundary events.
\item \textbf{Uncertain:} At least one window has $|\Delta\text{AICc}| < 2$, making classification ambiguous.
\end{itemize}

At the default threshold: \LIGOFlipDeterminatePct\% (\LIGOFlipDeterminateN/\LIGOFlip) of flips are determinate, \LIGOFlipUncertainPct\% (\LIGOFlipUncertainN/\LIGOFlip) are uncertain. This indicates most flip events represent genuine near-boundary dynamics rather than fit ambiguity. A multi-timescale recovery is exactly the kind of structure that produces window-dependent geometry; we therefore treat the flip population as consistent with boundary or compound dynamics rather than assuming label failure.

\begin{figure}[H]
\centering
\includegraphics[width=0.95\textwidth]{../figures/appendix/flip_aicc_gap_distribution.png}
\caption{AICc gap distribution by stability class. \textbf{Left:} Boxplot showing minimum $|\Delta\text{AICc}|$ across windows for each stability class. Flip events have smaller evidence gaps than stable events. \textbf{Right:} Density distributions confirming that flip events cluster in the low-evidence regime ($|\Delta\text{AICc}| < 4$) where classification boundaries are genuinely ambiguous. Horizontal lines mark ``strong'' ($\ge 4$) and ``very strong'' ($\ge 10$) evidence thresholds. This confirms that flips represent true boundary dynamics, not label failure.}
\label{fig:flip_aicc}
\end{figure}

\subsection{Curvature Baseline Robustness}
\label{app:baseline}

Curvature $b$ depends on the baseline estimate. We tested three methods:

\begin{center}
\begin{tabular}{lccc}
\toprule
\textbf{Baseline Method} & \textbf{AUC($b$)} & \textbf{95\% CI} & \textbf{Cliff's $\delta$} \\
\midrule
Final 10~ms median (default) & \LIGOCurvatureAUC & [\LIGOCurvatureAUClo, \LIGOCurvatureAUChi] & \LIGOCliffsDelta \\
Final 20~ms median & \LIGOCurvatureAUC & [\LIGOCurvatureAUClo, \LIGOCurvatureAUChi] & \LIGOCliffsDelta \\
Theil--Sen linear extrapolation & \LIGOCurvatureAUC & [\LIGOCurvatureAUClo, \LIGOCurvatureAUChi] & \LIGOCliffsDelta \\
\bottomrule
\end{tabular}
\end{center}

The curvature separation is identical (within reported precision) across all three baseline methods. This is expected because the baseline estimate affects only the vertical offset of $z(t)$, not its early-time curvature; the quadratic coefficient $b$ is insensitive to additive shifts. This rules out baseline artifacts as the source of the observed separation.

\subsection{Curvature Window Robustness}
\label{app:curvature-window}

To verify that the curvature separation is not an artifact of the 20~ms window choice, we computed $b$ over three fit intervals:

\begin{center}
\begin{tabular}{lccc}
\toprule
\textbf{Fit Interval} & \textbf{AUC($b$)} & \textbf{95\% CI} & \textbf{Sign Stable?} \\
\midrule
0--10~ms & \LIGOCurvSweepTenAUC & [\LIGOCurvSweepTenCILo, \LIGOCurvSweepTenCIHi] & \LIGOCurvSweepTenSign \\
0--20~ms (default) & \LIGOCurvSweepTwentyAUC & [\LIGOCurvSweepTwentyCILo, \LIGOCurvSweepTwentyCIHi] & \LIGOCurvSweepTwentySign \\
0--30~ms & \LIGOCurvSweepThirtyAUC & [\LIGOCurvSweepThirtyCILo, \LIGOCurvSweepThirtyCIHi] & \LIGOCurvSweepThirtySign$^\dagger$ \\
\bottomrule
\end{tabular}
\end{center}

\noindent $^\dagger$At 30~ms, both populations have negative median $b$, but the delayed population is less negative, maintaining strong discriminability (AUC = \LIGOCurvSweepThirtyAUC).

\medskip

The separation effect persists across all three intervals with AUC $>$ 0.9 in all cases. ``Sign stable'' indicates that the delayed population has positive median $b$ (accelerating early recovery) while the fast population has negative median $b$ (decelerating). Cross-window Spearman correlations are high ($\rho = \LIGOCurvSpearmanTenTwenty$ for 10 vs 20~ms, $\rho = \LIGOCurvSpearmanTwentyThirty$ for 20 vs 30~ms), confirming that curvature is a stable event-level property rather than a windowing artifact. The 20~ms default was pre-specified as a compromise: it spans approximately $2\tau$ for 26-qubit data ($\tau \sim 11$~ms) while capturing the early-time geometry in LIGO.

\subsection{Preprocessing Sensitivity}

The Hilbert envelope extraction includes a bandpass filter (10--500 Hz). To check whether this choice drives the curvature separation, we verified stability under alternative preprocessing:

\begin{itemize}
\item \textbf{Narrower bandpass (30--300 Hz):} Stable-core classification unchanged; AUC$(b)$ within 0.02 of default
\item \textbf{Wider bandpass (5--1000 Hz):} Stable-core classification unchanged; AUC$(b)$ within 0.02 of default
\item \textbf{No bandpass (raw envelope):} Higher noise floor; stable-core tallies within $\pm 3$ events of default; AUC$(b) = 0.87$
\end{itemize}

The curvature separation is robust to reasonable bandpass variations. The filter primarily affects noise floor, not the systematic difference between delayed and fast populations.

\subsection{Dip Test Without Stability Labels}

A potential concern is that the curvature bimodality is induced by conditioning on stability labels. To address this, we ran the Hartigan dip test on \textit{all} OK events without using stability labels.

\textit{Note on event counts:} Dip tests are computed only for unique GPS times with valid curvature estimates. After deduplication and exclusion of fit-failed events, this yields N=\LIGODipUnlabeledN\ for the pooled (stable + flip) cohort and N=\LIGODipStableN\ for stable-only.

\begin{center}
\begin{tabular}{lccc}
\toprule
\textbf{Cohort} & \textbf{N} & \textbf{Dip statistic} & \textbf{$p$-value} \\
\midrule
All OK (stable + flips) & \LIGODipUnlabeledN & \LIGODipUnlabeled & \LIGODipUnlabeledP \\
\quad + winsorized (1st--99th pctl) & -- & -- & \LIGODipUnlabeledWinsP \\
\quad + trimmed ($k=5$ each tail) & -- & -- & \LIGODipUnlabeledTrimP \\
Stable-only & \LIGODipStableN & -- & \LIGODipStableP \\
\quad + trimmed ($k=5$ each tail) & -- & -- & \LIGODipStableTrimP \\
\bottomrule
\end{tabular}
\end{center}

When flips are included, the dip test becomes non-significant ($p = \LIGODipUnlabeledP$), but this is expected: flips are explicitly \textit{boundary events} that occupy intermediate curvature values, filling in the gap between modes. The stable-only test remains significant (p = \LIGODipStableP). Winsorizing and trimming do not change the pattern, ruling out outlier-driven artifacts.

\subsection{Window-Level Regression Without Stability Conditioning}

The stable-only regression (Section~\ref{sec:ligo}) could be criticized for conditioning on stability---a form of collider bias. To address this, we ran a window-level logistic regression using \textit{all} determinate window classifications (not just stable events):

\begin{itemize}
\item \textbf{Unit of analysis:} The (event, window-length) pair; each event contributes up to 3 observations (60/100/150~ms windows), with SEs clustered by event ID
\item \textbf{Observations:} \LIGOWindowN\ window-level classifications from \LIGOWindowClusters\ unique events
\item \textbf{Outcome:} Delayed (1) vs.\ Fast (0) at each window
\item \textbf{Predictors:} Curvature $b$ (standardized), SNR (standardized)
\end{itemize}

\noindent\textbf{Results:} $\beta_b = +\LIGOWindowBetaB$ ($z = \LIGOWindowZB$, $p < 10^{-5}$); $\beta_{\text{SNR}} = \LIGOWindowBetaSNR$ ($p = \LIGOWindowPSNR$).

\noindent\textbf{Odds ratios:} OR$_b = \exp(1.61) \approx 5.0$ per $+1$ SD curvature; OR$_{\text{SNR}} = \exp(-0.31) \approx 0.73$ per $+1$ SD SNR. Higher curvature is strongly associated with delayed classification, while higher SNR is weakly associated with fast classification.

The curvature effect is highly significant ($p < 10^{-5}$) and survives SNR control, without conditioning on stability labels. This rules out collider bias as an explanation for the curvature--classification association.

\subsection{Unsupervised Validation: GMM on Curvature}
\label{app:gmm}

To verify that the curvature structure exists independently of model-derived labels, we fit a 2-component Gaussian Mixture Model (GMM) to curvature $b$ alone, ignoring classification labels entirely.

\begin{figure}[H]
\centering
\includegraphics[width=\textwidth]{../figures/appendix/unsupervised_gmm_validation.png}
\caption{Unsupervised GMM validation. \textbf{Left:} 2-component GMM fit to all OK events' curvature (ignoring labels), with component densities and decision boundary at $b = \LIGOGMMBoundary$. \textbf{Right:} Same curvature values colored by model-derived labels; flip events (gray) shown for context. Agreement metrics are computed on the stable-core subset only (Stable Delayed vs.\ Stable Fast); flips are excluded from ARI/NMI since they have no single ``ground truth'' label.}
\label{fig:gmm_validation}
\end{figure}

\noindent\textbf{Agreement metrics} (GMM clusters vs.\ model-derived stable labels; stable-core only):
\begin{itemize}
\item Agreement: \LIGOGMMAgreement\%
\item Adjusted Rand Index: \LIGOGMMARI
\item Normalized Mutual Information: \LIGOGMMNMI
\end{itemize}

The high agreement demonstrates that the curvature structure is not an artifact of the classification procedure: an unsupervised method recovers a similar two-population structure from the raw curvature data.

\subsection{Rejected-Morphology Stress Test}
\label{app:stress-test}

\noindent\textbf{Rejected-morphology stress test.}
To test whether the curvature--geometry association is an artifact of the single-pulse sanity filter, we re-ran the \emph{same} AICc tournament on events that failed the morphology criteria (multi-component peaks and/or non-decaying envelopes). These ``complex'' events are not used for population fractions because they typically lack 3-window stability; they are used only as an out-of-distribution robustness check.

Among \LIGORejectN\ complex events, the curvature distribution differs from the OK cohort (KS $D=\LIGORejectKS$, $p=\LIGORejectKSP$). Within the complex cohort, the tournament yields \LIGORejectDelayedN\ delayed and \LIGORejectFastN\ fast determinate classifications (with \LIGORejectUncertainN\ uncertain). Computing AUC$(b)$ on determinate labels only gives AUC$(b)=\LIGORejectAUC$, comparable to the OK stable-core AUC of \LIGOCurvatureAUC.

\noindent\emph{Interpretation:} the curvature--geometry association is also present in rejected morphologies, suggesting the association is not uniquely explained by the morphology filter; the filter's primary role here is to make single-pulse events analyzable and stabilize estimation. Because the rejected cohort has a different curvature distribution (KS), AUC values should be read qualitatively (association present/absent), not as directly comparable effect sizes.

\subsection{Alternative Likelihood: Log-Domain Fitting}
\label{app:alt-likelihood}

\noindent\textbf{Alternative likelihood (log-domain).}
Because Hilbert-envelope amplitudes are positive-valued and may exhibit multiplicative noise, we repeated the full 3-window tournament on $\log E(t)$.
Agreement with the original pipeline is \LIGOAltLikelihoodAgreement\%.
Notably, the Stable Fast count is unchanged (147 in both pipelines), and most disagreements (29/35) are Stable$\leftrightarrow$Flip rather than Delayed$\leftrightarrow$Fast.

\noindent\emph{Interpretation:} the delayed/fast geometry split is robust to likelihood misspecification; the main sensitivity is boundary assignment for near-threshold events, consistent with a genuine boundary population rather than a fragile regime classification.

\subsection{Bootstrap Seed Robustness}
\label{app:seed-robustness}

During development, we validated that results are stable across different random subsamples. We performed bootstrap resampling (N=926 events per resample) across 10 different random seeds: 1, 7, 23, 42, 59, 97, 127, 256, 500, and 999. The final analysis uses the full catalog.

\textbf{Protocol:}
\begin{itemize}
\item Same sample size per seed (N=926 events)
\item Same event selection rules; only RNG seed varies
\item All hyperparameters fixed (windowing, filtering, AICc thresholds)
\end{itemize}

\noindent These sweeps confirmed that classification rates are stable across random subsamples, supporting the reliability of the full-catalog analysis.

\begin{center}
\begin{tabular}{lcccc}
\toprule
\textbf{Metric} & \textbf{Original} & \textbf{Median} & \textbf{Range} & \textbf{Std} \\
\midrule
Stable delayed (\%) & 34.6 & 34.8 & [33.0, 35.6] & 0.79 \\
AUC$(b)$ & 0.952 & 0.950 & [0.939, 0.958] & 0.007 \\
GMM agreement (\%) & 88.3 & 88.3 & [85.6, 90.3] & 1.5 \\
\bottomrule
\end{tabular}
\end{center}

\begin{figure}[H]
\centering
\includegraphics[width=\textwidth]{../figures/appendix/seed_sweep_stability.png}
\caption{Bootstrap robustness across 10 random seeds. All key metrics (stable delayed fraction, curvature AUC, GMM agreement) remain stable under bootstrap resampling, with standard deviations $< 1\%$ for population fractions and $< 0.01$ for AUC. The original seed (42) produces results within the typical range, confirming that findings are not artifacts of a particular sample draw.}
\label{fig:seed_sweep}
\end{figure}

\noindent\emph{Interpretation:} The stable delayed fraction varies by only $\pm 1$--2\% across bootstrap resamples, and the curvature AUC remains $> 0.93$ in all cases. The seed-42 results are representative of the distribution. This rules out sample-specific artifacts as an explanation for the two-regime structure.

\subsection{Null Simulation Control (Window-Decoupled)}
\label{app:null-control}

To assess whether the observed ``delayed'' population could arise from analysis artifacts, we generated synthetic strain-envelope recoveries drawn exclusively from fast-geometry families (exponential and rational). Crucially, geometry labels were inferred using a \emph{late-time window only} (30--100~ms), while the curvature index was computed using an \emph{early-time window only} (0--20~ms), with baseline estimated from a disjoint tail window (110--150~ms). This window-decoupled design eliminates label$\to$curvature leakage by construction.

\paragraph{Null-fast-only simulation.}
In a pure-fast null world ($n = \LIGONullN$), we generated synthetic traces with fast-geometry (exponential or rational decay, randomly selected) and measured:
\begin{itemize}
\item \textbf{False-delayed rate:} \LIGONullMislabelRate\% (95\% Wilson CI: [\LIGONullMislabelCILo\%, \LIGONullMislabelCIHi\%])
\item \textbf{Ambiguous classification:} \LIGONullAmbiguousN\ events (lacking $\Delta$AICc $\geq 6$)
\end{itemize}

This false-delayed rate defines a conservative ``ceiling'' for classification noise. In the real LIGO sample, the delayed fraction is \LIGOStableDelayedPct\%---approximately 35$\times$ above the null ceiling.

\paragraph{Mixed-truth validation.}
To verify that curvature genuinely separates geometric classes (not just classification noise), we generated a mixed population: 50\% fast-geometry (exponential/rational) + 50\% delayed-geometry (delayed-exponential/sigmoid) with known ground truth labels.

Of \LIGOMixedNTotal\ traces, \LIGOMixedNValid\ (\LIGOMixedAccuracy\%) received clear classifications (fast or delayed); the remaining \LIGOMixedNAmbiguous\ were ambiguous (no model won by $\Delta$AICc $\geq 6$). The confusion matrix for assigned events:
\begin{center}
\begin{tabular}{lcc}
\toprule
& \textbf{Pred.\ Fast} & \textbf{Pred.\ Delayed} \\
\midrule
\textbf{True Fast} & \LIGOMixedCMFF & \LIGOMixedCMFD \\
\textbf{True Delayed} & \LIGOMixedCMDF & \LIGOMixedCMDD \\
\bottomrule
\end{tabular}
\end{center}
Classifier accuracy (excluding ambiguous) is 98.6\%. More importantly:
\begin{itemize}
\item \textbf{AUC (curvature vs.\ true labels):} \LIGOMixedAUCTrue
\item \textbf{AUC (curvature vs.\ assigned labels):} \LIGOMixedAUCAssigned
\end{itemize}

The near-unity AUC against \emph{true} labels confirms that early-time curvature genuinely separates fast from delayed geometry---the curvature signature is not circular with the classification procedure.

\paragraph{Parameter stress sweep.}
To stress-test the null ceiling, we swept noise scale ($\sigma \in \{0.01, 0.03, 0.05, 0.1\}$), AICc threshold ($\Delta \in \{4, 6, 10\}$), delay threshold ($\tau_{\min} \in \{0, 5, 10\}$~ms), and late-window bounds. Results are summarized in Table~\ref{tab:null-control}.

\begin{table}[h]
\centering
\caption{Null Control Summary: False-delayed rates under various configurations}
\label{tab:null-control}
\begin{tabular}{lccc}
\toprule
\textbf{Condition} & \textbf{Delayed \%} & \textbf{95\% CI Upper} & \textbf{Note} \\
\midrule
Null baseline (30--100ms) & \LIGONullMislabelRate & \LIGONullMislabelCIHi & Nominal pipeline \\
Worst-case (fixed window) & \LIGOSweepFixedMax & \LIGOSweepFixedMaxCIHi & Noise/AICc/delay sweep \\
Worst-case (any window) & \LIGOSweepWindowMax & \LIGOSweepWindowMaxCIHi & Window \LIGOSweepWorstWindow~ms \\
\midrule
\textbf{Real LIGO} & \textbf{\LIGOStableDelayedPct} & --- & Observed \\
\bottomrule
\end{tabular}
\end{table}

\noindent
Even under the most adversarial parameter tuning (window 40--120~ms), the null ceiling reaches only $\sim$12\%---still well below the observed \LIGOStableDelayedPct\%. Under nominal settings, the gap is $>$20$\times$. Figure~\ref{fig:null-control} visualizes this comparison.

\begin{figure}[ht]
\centering
\includegraphics[width=0.85\textwidth]{../figures/appendix/null_control_bars.pdf}
\caption{Null simulation control: false-delayed ceiling under various configurations versus real LIGO. The null baseline (30--100~ms window) produces only $\sim$1.5\% false-delayed classifications. Even under adversarial window tuning (40--120~ms), the ceiling reaches only $\sim$12\%---well below the observed \LIGOStableDelayedPct\% in real LIGO. Error bars show 95\% Wilson confidence intervals.}
\label{fig:null-control}
\end{figure}

\paragraph{Spliced-null control (cross-trace shuffle).}
The strongest test of window decoupling is the ``spliced-null'' control: we cross-match early-window curvature from one trace with late-window labels from a \emph{different} trace. If the high AUC were due to spurious within-trace correlation (e.g., overlapping windows), the spliced AUC would remain high. Instead:
\begin{itemize}
\item \textbf{Unspliced AUC} (curvature vs.\ same-trace labels): \LIGOSplicedAUCUnspliced
\item \textbf{Spliced AUC} (curvature vs.\ shuffled labels, 100 random shuffles): \LIGOSplicedAUCMean\ $\pm$ \LIGOSplicedAUCStd
\end{itemize}
The AUC drop of \LIGOSplicedAUCDrop\ (from \LIGOSplicedAUCUnspliced\ to chance level) after shuffling confirms that early-window curvature genuinely predicts late-window geometry---there is no pipeline-induced correlation.

\begin{center}
\fbox{\parbox{0.9\textwidth}{
\textbf{Null-Control Interpretation:} The null simulation establishes that (i)~the \LIGOStableDelayedPct\% delayed fraction in real LIGO \emph{cannot} be explained by pipeline mislabeling (ceiling $\leq$\LIGOSweepOverallMaxCIHi\% under any tested configuration), (ii)~the curvature index is a valid discriminator when ground truth is known, and (iii)~window decoupling is complete (spliced AUC = 0.50). \textbf{The delayed population is real, not a classification artifact.}
}}
\end{center}

% ===== END INLINED: appendices.tex =====

\end{document}
